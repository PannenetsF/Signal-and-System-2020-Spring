\ifx\mainclass\undefined
\documentclass[cn,11pt,chinese,black,simple]{../elegantbook}
\usepackage{array}
\newcommand{\ccr}[1]{\makecell{{\color{#1}\rule{1cm}{1cm}}}}


\newcommand{\where}[1]{\Big|_{#1}}
\newcommand{\dd}[1]{\mathrm{d}#1}
\newcommand{\abs}[1]{\left| #1 \right|}
\newcommand{\zt}[1]{\mathscr{Z}[#1]}
\newcommand{\zta}{\xrightarrow{\mathscr{Z}}} 
\newcommand{\lt}[1]{\mathscr{L}[#1]}
\newcommand{\lta}{\xrightarrow{\mathscr{L}}} 
\newcommand{\ft}[1]{\mathscr{F}[#1]}
\newcommand{\fta}{\xrightarrow{\mathscr{F}}} 
\newcommand{\dsum}{\displaystyle\sum}
\newcommand{\aint}{\int_{-\infty}^{+\infty} }

\newcommand{\re}{\operatorname{Re}[\mathrm{s}]} 


\newcommand{\qfig}[2]{\begin{figure}[!htb]
  \centering
  \includegraphics[width=0.6\textwidth]{#1}
  \caption{#2}
\end{figure}}



\renewcommand\arraystretch{1.6}




\setcounter{tocdepth}{3}
\newcommand{\dollar}{\mbox{\textdollar}}
\lstset{
  mathescape = false}

  

\usepackage{shapepar}
\usepackage{longtable}
\usepackage{tikz}
% \usepackage{multirow}
\usetikzlibrary{positioning}
\tikzset{>=stealth}
\newcommand{\tikzmark}[3][]
  {\tikz[remember picture, baseline]
    \node [anchor=base,#1](#2) {#3};}
\usetikzlibrary{graphs}
\begin{document}
\fi 

% Start Here

\chapter{信号与系统概论}



本章讨论信号与系统的基础概念问题。

\section{信号的分类}

\begin{definition}[能量/功率信号]\label{def:ene-powersig}
  对于信号 \(f(t)\) , 定义其能量为 : 
  \[W = \lim_{T\rightarrow \infty} \int_{-T/2}^{T/2} f^2(t) \dd{t} \]
  定义其功率为 : 
  \[P = \lim_{T\rightarrow \infty} \dfrac{1}{T} \int_{-T/2}^{T/2} f^2(t) \dd{t} \] 
\end{definition}

\section{信号的变换}

掌握信号的压缩、扩展、反褶以及时移等运算。

\section{基本信号}

本书涉及到的基本信号可以列举如下。

\begin{itemize}
  \item 指数类信号 
    \begin{itemize}
      \item 实指数信号 \(K e^{\alpha t}\)
      \item 复指数信号 \(K e^{(\sigma + j \omega)t}\)
      \item 正余弦信号 \(\cos(\omega t + \varphi)\), \(\sin(\omega t + \varphi)\)
    \end{itemize}
  \item 奇异信号
    \begin{itemize}
      \item 单位冲激信号 \(\delta(t)\)
      \item 单位阶跃信号 \(u(t)\)
      \item 单位斜变信号 \(r(t)\)
    \end{itemize}
\end{itemize}

\section{系统的描述}

本课程研究的系统均可用微分方程或者差分方程描述,这样的系统可以使用框图进行描述。

系统可以使用以下标准进行分类。

\begin{itemize}
  \item 连续时间系统与离散时间系统
  \item 有记忆系统与无记忆系统
  \item 可逆系统与不可逆系统 : 不同的激励是否会引起不同的相应
  \item 因果系统与非因果系统 : 响应出现不早于激励
  \item 稳定系统与不稳定系统 : 有限输入是否引起有限输出
  \item 线性系统与非线性系统
  \item 时变系统与时不变系统 : 输入的时移对应引起输出的时移
\end{itemize}



% End Here

\ifx\mainclass\undefined
\end{document}
\fi 