\ifx\mainclass\undefined
\documentclass[cn,11pt,chinese,black,simple]{../elegantbook}
\usepackage{array}
\newcommand{\ccr}[1]{\makecell{{\color{#1}\rule{1cm}{1cm}}}}


\newcommand{\where}[1]{\Big|_{#1}}
\newcommand{\dd}[1]{\mathrm{d}#1}
\newcommand{\abs}[1]{\left| #1 \right|}
\newcommand{\zt}[1]{\mathscr{Z}[#1]}
\newcommand{\zta}{\xrightarrow{\mathscr{Z}}} 
\newcommand{\lt}[1]{\mathscr{L}[#1]}
\newcommand{\lta}{\xrightarrow{\mathscr{L}}} 
\newcommand{\ft}[1]{\mathscr{F}[#1]}
\newcommand{\fta}{\xrightarrow{\mathscr{F}}} 
\newcommand{\dsum}{\displaystyle\sum}
\newcommand{\aint}{\int_{-\infty}^{+\infty} }

\newcommand{\re}{\operatorname{Re}[\mathrm{s}]} 


\newcommand{\qfig}[2]{\begin{figure}[!htb]
  \centering
  \includegraphics[width=0.6\textwidth]{#1}
  \caption{#2}
\end{figure}}



\renewcommand\arraystretch{1.6}




\setcounter{tocdepth}{3}
\newcommand{\dollar}{\mbox{\textdollar}}
\lstset{
  mathescape = false}

  

\usepackage{shapepar}
\usepackage{longtable}
\usepackage{tikz}
% \usepackage{multirow}
\usetikzlibrary{positioning}
\tikzset{>=stealth}
\newcommand{\tikzmark}[3][]
  {\tikz[remember picture, baseline]
    \node [anchor=base,#1](#2) {#3};}
\usetikzlibrary{graphs}
\begin{document}
\fi 

% Start Here
\chapter{连续时间系统的实频域分析}

\section{系统频率响应}

\subsection{定义与意义}

\begin{definition}[系统频率响应]
    系统的单位冲激响应 \(h(t)\) 的 傅里叶变换 \(H(\omega)\) 为系统频率响应,又称系统的传递函数。

    \[h(t) \fta H(\omega)\]
\end{definition}

根据卷积定理,激励信号 \(e(t)\) 作用到系统的零状态响应 \(r(t)\) 的频域形式为 

\[R(\omega) = E(\omega) H(\omega)\]

以基本信号\(\sin(\omega_1 t + \theta_1)\) 为例,分析系统对不同的频率信号的影响

\[R(\omega) = -j \pi \left[\delta(\omega - \omega_1) - \delta(\omega + \omega_1)\right] e^{ j \displaystyle\dfrac{\theta_1}{\omega_1} \omega} H(\omega) \]

\[r(t) = \dfrac{1}{2 j} \left[H(\omega_1) e^{j \omega_1 t + j \theta_1} - H(-\omega_1) e^{-j \omega_1 t - j\theta_1}\right]\]

分解 \[H(\omega) = |H(\omega)|e^{j \varphi(\omega)}\]

得到 \[r(t) = |H(\omega_1) sin\left[\omega_1 t + \theta_1 + \varphi(\omega_1)\right]\]

即特定频率信号通过 \(H(\omega)\) 的系统, 幅度会被 \(|H(\omega)|\) 调制, 并且发生 \(\varphi(\omega)\) 的相移。

\subsection{求解}

可以通过对系统的常微分方程做傅里叶变换的方式求得系统频响。


\[
\dsum_{i=0}^{n} C_i \dfrac{\dd{^{n-i} r(t)}}{\dd{t^{n-i}}} = \dsum_{j=0}^{m} E_j \dfrac{\dd{^{m-j} e(t)}}{\dd{t^{m-j}}} 
\]

\[ H(\omega) = \dfrac{\dsum_{i = 0}^{n-1} C_i(j \omega)^{n - i}}{\dsum_{j = 0}^{m-1} E_i (j \omega)^{m-j}} \]


\section{无失真系统}

\begin{definition}[无失真系统]
    系统的输出可以复现系统输入信号,区别仅在于出现的时刻与幅度,信号的形状并未发生改变,这样的系统称为无失真系统。
    即
    \[r(t) = k e(t - t_0)\]
\end{definition}

\subsection{系统频响}

分析系统的单位冲激响应

\[h(t) = k \delta(t - t_0)\]

那么

\[H(\omega) = k e^{-j \omega t_0}\]

因此幅频特性与相频特性分布为

\begin{equation*}
    \begin{aligned}
        \abs{H(\omega)} &= k \\
        \varphi(\omega) &= - \omega t_0
    \end{aligned}
\end{equation*}

\subsection{失真类型}

幅度失真,对不同频率分量的幅度加权不一致 \[\abs{H(\omega_1)} \neq \abs{\omega_2}\]

相位失真,对不同频率分量的相移不一致 \[\dfrac{\varphi(\omega_1)}{\omega_1} \neq \dfrac{\varphi(\omega_2)}{\omega_2}\]

\begin{definition}[群延时]
    相频特性对频率的负导数 
\end{definition}

无失真系统的群延时的物理意义是信号的延时时间。

存在相位失真但是幅度不失真的系统称为\textbf{全通网络}或者\textbf{全通系统}。

\section{理想低通滤波器}

\subsection{频响特性}

对特定频率以下的信号无损滤波,应当满足

\begin{equation*}
    H(\omega) = 
    \left\{
        \begin{aligned}
            &e^{-j \omega t_0}, \abs{\omega} < \omega_c\\
            &0, \abs{\omega} \geq \omega_c
        \end{aligned}
    \right.
    \rightarrow
    \left\{
    \begin{aligned}
        \abs{H(\omega)} &= \left\{ 
            \begin{aligned}
                &1, \abs{\omega} < \omega_c \\ 
            &0 , \abs{\omega} \geq \omega_c     
            \end{aligned}
        \right. \\
        \varphi(\omega) &= \left\{
            \begin{aligned}
                & -\omega t_0, \abs{\omega} < \omega_c\\
                & 0, \abs{\omega} \geq \omega_c
            \end{aligned}
        \right.
    \end{aligned}
    \right.
\end{equation*}

进行逆变换,得到其时域波形

\[h(t) = \dfrac{\omega_c}{\pi} Sa\left[\omega_c(t-t_0)\right]\]

可以发现理想低通滤波器是非因果且失真的。

\subsection{单位阶跃响应}

\begin{definition}[\(Si\) 函数]
    \[Si(t) = \int_0^t \dfrac{\sin t }{t} \dd{t}\]
\end{definition}

理想低通滤波器的单位阶跃响应为 
\[G(\omega) = \left[\pi \sigma(\omega) + \dfrac{1}{j \omega} \right] e^{-j \omega t_0}, -\omega_c < \omega < \omega_c\]

那么进行逆变换

\begin{equation*}
    \begin{aligned}
        g(t) &= \dfrac{1}{2 \pi} \int_{-\infty}^{+\infty} G(\omega) \dd{\omega}\\
        &= \dfrac{1}{2} + \dfrac{2}{2 \pi}  \int_0^{\omega_c} \dfrac{\sin \omega (t-t_0)}{\omega} \dd{\omega}\\
        &= \dfrac{1}{2} + \dfrac{1}{\pi} \int_0^{\omega_c (t-t_0) \dfrac{\sin x}{x} \dd{x}}
    \end{aligned}
\end{equation*}

定义上升时间为从最小值到最大值的时间,那么阶跃响应的上升时间为 \[t_r = \dfrac{2 \pi }{\omega_c}\] 上升时间与带宽成反比。


\subsection{矩形脉冲响应}

吉布斯现象指的是矩形波的上冲现象。从一般的理解来看,随着滤波器截止频率的增大,通过的
频率分量增多,这个峰值应该逐渐减小。
但实际情况是,只有当系统的截止频率为无穷大时上冲才
会消失,否则截止频率即使再大其依然存在,且大约等于
总跳变值的\(9 \% \)。

\section{系统的因果性}

因果系统的单位冲激响应满足\[h(t)  = h(t) u(t)\]

\begin{theorem}[佩利-维纳准则]
对于幅频特性 \(\abs{H(\omega)}\) 系统物理可以实现的必要条件是 

\[\int_{-\infty}^{+\infty} \dfrac{\abs{\ln \abs{H(\omega)}}}{1 + \omega^2} \dd{\omega} < \infty\]

且 \(\abs{H(\omega)}\) 平方可积

\[\aint \abs{H(\omega)}^2 \dd{\omega} < \infty\]

\end{theorem}

值得注意的是,若在一段有限长度的区间上幅频特性为  \(0\), 那么\(\ln \abs{H(\omega)}\) 为无穷,因此不满足定理,无法物理实现。所以理想低通滤波器、理想高通滤波器、理想带通滤波器和理想带阻滤波器都是物理不可实现的系统。


\begin{theorem}[希尔伯特变换]
    系统响应划分为实部虚部
    \[H(\omega) = R(\omega) + j X(\omega)\]
    那么频域判定LTI为因果稳定系统的充分必要条件是,系统频率响应的实部和虚部满足
    \begin{equation*}
        \begin{aligned}
            R(\omega) &= X(\omega) \otimes \dfrac{1}{\pi \omega}\\
            X(\omega) &= R(\omega) \otimes (- \dfrac{1}{\pi \omega})\\
        \end{aligned}
    \end{equation*}
\end{theorem}

推导如下 对 \(h(t) = h(t) u(t)\) 进行傅里叶变换

\[ H(\omega) = \dfrac{1}{2 \pi} H(\omega) \otimes \left[\pi \sigma(\omega) + \dfrac{1}{j \omega}\right] \]

分解

\[H(\omega) = R(\omega) + j X(\omega)\]

带入

\begin{equation*}
    \begin{aligned}
        R(\omega) + j X(\omega) &= \dfrac{1}{2 \pi} \left[R(\omega) + j X(\omega) \right] \otimes \left[\pi \sigma(\omega) + \dfrac{1}{j \omega} \right] \\
        &= \dfrac{1}{2\pi}\left[\pi R(\omega) + X(\omega)\otimes \dfrac{1}{\omega}\right] + \dfrac{j}{2 \pi} \left[\pi X(\omega) - R(\omega) \otimes \dfrac{1}{\omega}\right]
    \end{aligned}
\end{equation*}

对应的实部虚部相等

\begin{equation*}
    \begin{aligned}
        R(\omega) &= X(\omega) \otimes \dfrac{1}{\pi \omega}\\
        X(\omega) &= R(\omega) \otimes (- \dfrac{1}{\pi \omega})
    \end{aligned}
\end{equation*}

以上两式分别称为希尔伯特正变换、逆变换,分别相当于通过 \(h(t) = \dfrac{1}{\pi t}\) 与 \(h(t) = -\dfrac{1}{\pi t}\) 的系统。

希尔伯特正变换:

\[\hat{f}(t) = f(t) \otimes \dfrac{1}{\pi t}\]

而 \[\dfrac{1}{\pi t} \fta e^{-j\dfrac{\pi}{2}} sgn \omega \]

对正频率部分相位移动\(-\pi/2\) 负频率部分移动 \(\pi / 2\)

实际上掌握概念即可,这里 \(\omega, t\) 的使用\textbf{没有时域和频域必然联系}。


\section{相关函数}

在之前的章节,定义了相关系数来反映信号的相似性,但是在信号传输过程中可能会出现时移,所以静态的相关系数无法全面反映相似关系。

\begin{definition}[相关函数(能量)]
    定义能量有限信号 \(f_1(t)\) 与 \(f_2(t)\) 的相关函数:
    \[R_{12}(\tau) = \aint f_1(t) f_2^*(t-\tau) \dd{t}\]

    \[R_{21}(\tau) = \aint f_2(t) f_1^*(t-\tau) \dd{t}\]
    
    函数与自身的相关函数称为自相关函数。
\end{definition}

易知,\(R_{12} = R_{21}^*\) ,并且可以观察到和卷积的相似性

\[f_i(t) \otimes f_j(-t) = R_{ij}(t)\]

在频域中 \(f_1(t) \fta F_1(\omega)\) , \(f_2(t) \fta F_2(\omega)\) 那么

\[\ft{R_{12}(\tau)} = F_1(\omega) F_2(-\omega) = F_1(\omega) F_2^*(\omega)  \]


\begin{definition}[相关函数(功率)]
    定义功率有限信号 \(f_1(t)\) 与 \(f_2(t)\) 的相关函数:
    \[R_{12}(\tau) = \lim_{T\rightarrow\infty}\dfrac{1}{T}\int_{-T/2}^{T/2} f_1(t) f_2^*(t-\tau) \dd{t}\]

    \[R_{21}(\tau) = \lim_{T\rightarrow\infty}\dfrac{1}{T}\int_{-T/2}^{T/2} f_2(t) f_1^*(t-\tau) \dd{t}\]
    
    函数与自身的相关函数称为自相关函数。
\end{definition}


\section{激励响应的谱关系}


\subsection{能量谱}

从自相关函数入手,可以得到

\[f(t) \otimes f(-t) \fta F(\omega) F(-\omega) = \abs{F(\omega)}^2\]

\(\tau = 0\) 时有 

\begin{equation*}
    \begin{aligned}
        R(0) &= \dfrac{1}{2 \pi} \aint \abs{F(\omega)}^2 \dd{\omega} \\
        &= \aint \abs{f(t)}^2 \dd{t} \\
        &= \aint \abs{F(2 \pi f)}^2 \dd{f}\\
    \end{aligned}
\end{equation*}

上式体现了能量守恒定律在时域与频域的体现,称为帕塞瓦尔方程或者能量守恒方程。将 \(\epsilon(\omega) = \abs{F(\omega)}^2 \) 称为信号的能量谱密度,是 \(\omega\) 的偶函数,它只决定于信号 \(f(t)\) 的频谱函数的模,
而与相位无关,其单位是焦耳\(\cdot\)秒。同时与自相关函数构成变换对

\[R(t) \fta \epsilon(\omega)\]


\subsection{功率谱}

仿照能量谱的方式,通过去极限的方式进行处理。截取某区间的功率信号

\[f_T(t) = f(t) \left[u(t+\dfrac{T}{2}) - u(t - \dfrac{T}{2})\right]\]

其能量表示为

\[E_T = \aint f_T^2(t) \dd{t} = \dfrac{1}{2 \pi} \aint \abs{F_T(\omega)}^2 \dd{\omega}\]

那么功率为 

\[P = \lim_{T\rightarrow \infty} \dfrac{1}{T} E_T = \dfrac{1}{2\pi}\aint  \lim_{T\rightarrow \infty} \dfrac{\abs{F_T(\omega)}^2}{T} \dd{\omega}\]

那么在 \(\lim_{T\rightarrow \infty}\dfrac{\abs{F_T(\omega)}^2}{T}\) 存在且有限时,定义功率谱密度函数为

\[p(\omega) = \lim_{T\rightarrow \infty} \dfrac{\abs{F_T(\omega)}^2}{T}\]

功率可以表示为 \(P = \dfrac{1}{2\pi}\aint p(\omega) \dd{\omega}\)

可以得到变换对,或维纳-欣钦定理

\[R(t) \fta p(\omega)\]

\subsection{LTI系统激励与响应的谱关系}

对功率或能量信号均满足输出谱为输入谱和系统频响模方的乘积。

\begin{equation*}
    \begin{aligned}
        p_r(\omega) &= \abs{H(\omega)}^2 p_e(\omega) \\
        \epsilon_r(\omega) &= \abs{H(\omega)}^2 \epsilon_e(\omega) 
    \end{aligned}
\end{equation*}

在时域进行分析

\[R(\omega)R(-\omega) = E(\omega)E(-\omega)H(\omega)H(-\omega)\]

\[R_r(\tau) = R_e(\tau) \otimes h(t) \otimes h(-t) = R_e(\tau) \otimes R_h(\tau)\]

定义系统冲激响应的自相关函数 \(R_h(\tau) = \otimes h(t) \otimes h(-t)\)

\section{实用性抽样系统分析模型}

一般通过理想抽样脉冲调制之后通过抽样系统获得抽样信号:

\[t: f_s(t) = [(f(t) \delta_{T_s}(t))] \otimes h_s(t) \]

\[\omega: F_s(\omega) = F_S^\delta(\omega) H_s(\omega) = \dfrac{H_s(\omega)}{T_s}\dsum_{n = -\infty}^{+\infty} F(\omega - n \omega_s)\]

那么通过低通滤波器回复原信号

\[H(\omega) = \dfrac{T_s}{H_s(\omega)}\]

\subsection{抽样保持信号}

实际上就是选取不同形状抽样信号。

零阶抽样保持:通过矩形脉冲采样 \(u(t)-u(t-T_s)\)

一阶抽样保持:通过尖峰脉冲采样 \(1 - \dfrac{\abs{t}}{T_s}\)

\subsection{幅度调制与解调}

调制是指通过特定信号的某种特性按照所需传输信号的
变化规律变化,该特定信号称为载波,所需传输的信号
称为调制波。

解调是调制的逆过程,将需传输的信号从接收到的信号
中提取出来。

若一个信号通过余弦信号调制

\[f_c(t) = f(t) \cos \omega_c t\]

\[F_c(\omega) = \dfrac{1}{2} [F(\omega-\omega_c) - F(\omega + \omega_c)]\]

幅度解调的实现方法和调制相近, 首先,将接收信号\(f_s(t)\)与接收端的本地载波信号相乘, 这里要
求本地载波信号是与发送端同频率同相位的余弦信号, 这种解调方式称为
同步解调。可以将频谱移到原点附近,直接滤波即可。

% End Here

\ifx\mainclass\undefined
\end{document}
\fi 