\ifx\mainclass\undefined
\documentclass[cn,11pt,chinese,black,simple]{../elegantbook}
\usepackage{array}
\newcommand{\ccr}[1]{\makecell{{\color{#1}\rule{1cm}{1cm}}}}


\newcommand{\where}[1]{\Big|_{#1}}
\newcommand{\dd}[1]{\mathrm{d}#1}
\newcommand{\abs}[1]{\left| #1 \right|}
\newcommand{\zt}[1]{\mathscr{Z}[#1]}
\newcommand{\zta}{\xrightarrow{\mathscr{Z}}} 
\newcommand{\lt}[1]{\mathscr{L}[#1]}
\newcommand{\lta}{\xrightarrow{\mathscr{L}}} 
\newcommand{\ft}[1]{\mathscr{F}[#1]}
\newcommand{\fta}{\xrightarrow{\mathscr{F}}} 
\newcommand{\dsum}{\displaystyle\sum}
\newcommand{\aint}{\int_{-\infty}^{+\infty} }

\newcommand{\re}{\operatorname{Re}[\mathrm{s}]} 


\newcommand{\qfig}[2]{\begin{figure}[!htb]
  \centering
  \includegraphics[width=0.6\textwidth]{#1}
  \caption{#2}
\end{figure}}



\renewcommand\arraystretch{1.6}




\setcounter{tocdepth}{3}
\newcommand{\dollar}{\mbox{\textdollar}}
\lstset{
  mathescape = false}

  

\usepackage{shapepar}
\usepackage{longtable}
\usepackage{tikz}
% \usepackage{multirow}
\usetikzlibrary{positioning}
\tikzset{>=stealth}
\newcommand{\tikzmark}[3][]
  {\tikz[remember picture, baseline]
    \node [anchor=base,#1](#2) {#3};}
\usetikzlibrary{graphs}
\begin{document}

\fi 



% Start Here
\chapter{连续时间信号的实频域分析}

本章在时域对信号进行分解进行研究,主要工具是傅里叶级数、傅里叶变换。这一章最重要的是掌握傅里叶变换的性质,进而大大简化计算过程。

\section{相关系数以及正交函数}

\begin{definition}[相关系数]
  记误差函数\(e(t)\)为\[e(t) = f(t) - k g(t)\]
  其均方误差为\[\bar{e^2(t)} = \int_{t_1}^{t_2}[f(t)-k g(t)]^2 \dd{t}\]
  在均方误差取得最小值时的 \(k\) 取值即为 \(f(t)\), \(g(t)\) 的相关系数,表征两者的相似程度
  \[k = \dfrac{\int_{t_1}^{t_2} f(t) g(t) \dd{t}}{\int_{t_1}^{t_2} g^2(t) \dd{t}}\]
\end{definition}

当两个函数的相关系数为 \(0\) 时,称这两个函数正交。

信号自身的正交分解常见两种:

\begin{itemize}
  \item 直流-交流分解 
  \item 奇偶分量分解
\end{itemize}

\section{三角函数形式傅里叶级数}

三角函数形式的傅里叶级数:

\[f(t) = a_0 + \dsum_{n=1}^{\infty} (a_n \cos n\omega_1 t + b_0 \sin \omega_1 t)\]

其中
\[a_0 = \dfrac{1}{T_1} \int_{-T_1/2}^{T/2} f(t) \dd{t}\]
\[a_n = \dfrac{2}{T_1} \int_{-T_1/2}^{T/2} f(t) \cos n \omega_1 \dd{t}\]
\[b_n = \dfrac{2}{T_1} \int_{-T_1/2}^{T/2} f(t) \sin n \omega_1 \dd{t}\]

三角函数形式的频谱分为幅度谱以及相位谱。幅度谱为
\[c_n = \sqrt{a_n^2 + b_n^2}\]

相位谱为
\[\phi_n = \tan^{-1}\left(\dfrac{-b_n}{a_n}\right)\]

\section{指数函数形式傅里叶级数}

指数形式的傅里叶级数:

\[f(t) = \dsum_{n=-\infty}^{+\infty} F(n\omega_1) e^{j n \omega_1 t}\]

其中
\[F(n\omega_1) = \dfrac{1}{T_1} \int_{-T_1/2}^{T_1/2} f(t) e^{-j n \omega_1 t} \dd{t}\]

根据三角函数形式级数的运算可以得到

\[a_{-n} = a_n, b_{-n} = -b_n\]

那么有

\[F(n \omega_1) = \left\{
\begin{aligned}
  & a_0, \text{n = 0}\\
  & \dfrac{a_n - j b_n}{2}, \text{n \(\neq\) 0}
\end{aligned}\right.
\]

称\(F{n\omega_1}\)为信号的复数频谱或者复数谱。

类似的,频谱为
\[ F(n \omega_1) = |F(n \omega_1)| e^{j \varphi_n} \]

\section{三角函数-指数函数形式傅里叶级数的频谱关系}

三角函数幅度谱与相位谱均为单边谱,对于幅度谱为偶拓延

\[ 
|F(n \omega_1)| = \left\{\begin{aligned}
  c_0 = a_0, n = 0 \\
  \dfrac{1}{2} \sqrt{a_n^2 + b_n^2} = \dfrac{1}{2} c_n , n \neq 0
\end{aligned}\right\}
\]

对于相位谱为奇拓延

\[
\varphi_n = \arctan (\dfrac{-b_n}{a_n})  
\]

\section{对称性}

\begin{longtable}{lll} 
  \caption{傅里叶级数的奇偶对称性} \\ 
  \toprule
  函数分类 & 三角级数性质 & 指数级数性质\\ 
  \midrule
  \endfirsthead
  
  \toprule
  函数分类 & 三角级数性质 & 指数级数性质\\ 
  \midrule
  \endhead 

  \hline
  \multicolumn{3}{c}{见下页}\\   \bottomrule
  \endfoot

  \bottomrule
  \endlastfoot
  奇函数                                                                              & 只包含正弦函数分量                                                      & \(F(n \omega_1)\) 为纯虚数 相位谱只取\(\pm \dfrac{1}{2} \pi\) \\ 
  偶函数                                                                              & 只包含余弦函数分量                                                      & \(F(n \omega_1)\) 为实数 相位谱只取\(\pm  \pi\)             \\ 
  \begin{tabular}[c]{@{}l@{}}奇谐函数\footnote{定义为\(f(t) = -f(t\pm \dfrac{T}{2})\)}\\\end{tabular} & \begin{tabular}[c]{@{}l@{}}只含有奇次谐波,\\ 不含偶次谐波与直流分量\end{tabular} &                                                     \\
  \begin{tabular}[c]{@{}l@{}}偶谐函数\footnote{定义为\(f(t) = f(t\pm \dfrac{T}{2})\)}\\\end{tabular}  & \begin{tabular}[c]{@{}l@{}}只含有次谐波与直流分量,\\ 不含有奇次谐波\end{tabular} &                    

\end{longtable}

\section{傅里叶变换}

\begin{definition}[傅里叶变换]
  对于任意的函数\(f(t)\),将其频谱密度函数
  \[F(\omega) = \int_{-\infty}^{+\infty} f(t) e^(-j\omega t) \dd{t}\]
  定义为傅里叶变换
\end{definition}
\section{傅里叶变换的性质}

假定已知变换为

\[f(t) \fta F(\omega)\]

性质列表如下


\begin{longtable}{ll} 
  \caption{傅里叶变换性质} \\ 
  \toprule
  名称 & 表达形式 \\ 
  \midrule
  \endfirsthead
  
  \toprule
  名称 & 表达形式 \\ 
  \midrule
  \endhead 

  \hline
  \multicolumn{2}{c}{见下页}\\   \bottomrule
  \endfoot

  \bottomrule
  \endlastfoot

  对称性 & \( F(t) \fta 2 \pi f(-\omega) \) \\
  奇偶虚实性 & 同傅里叶级数 \\
  反褶性质 & \(f(-t) \fta F(-\omega) = F^*(\omega)\) \\
  时移性质 & \(f(t-t_0) \fta F(\omega) e^{-j \omega t_0}\) \\
  压扩性质 & \(f(at) \fta \dfrac{1}{|a|}F(\dfrac{\omega}{a})\) \\
  线性性 &  \\
  时域微分 & \(f'(t) \fta j\omega F(\omega)\) \\
   & \(f^{(n)}(t) \fta (j\omega)^{(n)} F(\omega)\) \\
  时域积分 & \(\int_{-\infty}^{t}f(\tau) \dd{\tau} \fta \dfrac{F(\omega)}{j\omega} + \pi F(0) \delta(\omega)\) \\
  频移特性 & \(f(t)e^{j \omega_0 t} \fta F(\omega - \omega_0)\) \\
  卷积定理 & \(f_1(t) \otimes f_2(t) \fta F_1(\omega)F_2(\omega)\)\\
  & \(f_(t)f_1(t) \fta \dfrac{1}{2\pi} F_1(\omega)\otimes F_2(\omega)\)
  
  
\end{longtable}

\section{周期信号的傅里叶变换}

对于周期为 \(T\) 的信号 \(f(t)\) 有傅里叶级数


\[f(t) = \dsum_{n=-\infty}^{+\infty} F(n\omega_1) e^{j n \omega_1 t}\]

其中
\[F(n\omega_1) = \dfrac{1}{T_1} \int_{-T_1/2}^{T_1/2} f(t) e^{-j n \omega_1 t} \dd{t}\]

对其单周期信号 \(f_1(t)\) 有

\[f_1(t) = f(t) \left[u(t+\dfrac{T_1}{2}) - u(t-\dfrac{T_1}{2})\right]\]

其傅里叶变换 \(F_1(\omega)\) 与原函数的傅里叶级数满足

\[F(n \omega_1) = \dfrac{1}{T_1}F_1(\omega)\left|_{\omega = n \omega_1}\right.\]

带入傅里叶级数中得到

\[f(t) \fta 2\pi \dsum_{-\infty}^{+\infty} F(n\omega_1) \delta(\omega - n\omega_1)\]

或者可以简单地通过卷积表示。单个周期信号为 \(f_0(t)\) 那么

\[f_{T_1}(t) = f_0(t) \otimes \delta_{T_1}(t)\]

\[\ft{f_{T_1}(t)} = F_0(\omega) \dfrac{1}{T_1}\dsum_{-\infty}^{\infty} 2\pi \delta(\omega - n\omega_1) = F_0(\omega) \omega_1 \dsum_{-\infty}^{\infty} \delta(\omega - n\omega_1) \] 


\section{常见信号的傅里叶变换对}

\begin{definition}[$Sa$ 函数]
\[Sa(t) = \dfrac{\sin t}{t}\]
\end{definition}

在数字信号处理和通信理论中,归一化 \(sinc(x)\) 函数通常定义为 \(Sa(\pi x)\) 。

\begin{longtable}{ll} 
  \caption{傅里叶变换对} \\ 
  \toprule
  时域信号 & 频域信号 \\ 
  \midrule
  \endfirsthead
  
  \toprule
  时域信号 & 频域信号 \\ 
  \midrule
  \endhead 

  \hline
  \multicolumn{2}{c}{见下页}\\   \bottomrule
  \endfoot

  \bottomrule
  \endlastfoot

  \(G_{\tau}(t) = u(t+\dfrac{\tau}{2}) - u(t - \dfrac{\tau}{2})\) & \(\tau Sa(\dfrac{\omega \tau}{2})\) \\ 
  \(G_{\tau}(t) \otimes G_{\tau}(t) = \left\{\begin{aligned}
      \tau - |t|, |t| < \tau \\
      0, t \geq \tau
  \end{aligned}\right. \) & \(\tau^2 Sa^2(\dfrac{\omega \tau}{2})\) \\
  \(e^{-\alpha t}u(t)\) & \(\dfrac{1}{\alpha + j \omega}\) \\
  \(e^{-\alpha |t|}\) & \(\dfrac{2\alpha}{\alpha ^2 + \omega^2}\) \\
  \(sgn(t)\) & \(\dfrac{2}{j \omega}\) \\
  \(\delta(t)\) & \(1\) \\
  \(1\) & \(2 \pi \delta(\omega)\) \\
  \(u(t)\) & \(\pi \delta(\omega) + \dfrac{1}{j \omega}\) \\
  \(\sin \omega_1 t\) & \(-j \pi[\delta(\omega-\omega_1) -\delta(\omega + \omega_1)]\) \\
  \(\cos \omega_1 t\) & \(\pi [\delta(\omega-\omega_1) + \delta(\omega+\omega_1)]\) \\
  \(\delta_{T_1}(t) = \dsum_{-\infty}^{\infty}\delta(t-nT_1) \) & \(\omega_1 \dsum_{-\infty}^{\infty} \delta(\omega - n \omega_1)\) \\
  \(G_{T_1,\tau}(t)\) & \(2\pi \dsum_{-\infty}^{\infty} \dfrac{\tau}{T_1} Sa(\dfrac{n \omega_1 \tau}{2} ) \delta(\omega - n\omega_1)\) \\
  
  
\end{longtable}

\section{抽样信号}

原始信号为 \(f(t)\) ,抽样脉冲信号为 \(p(t)\) 

\[f_s(t) = f(t) p(t)\]

抽样信号为周期信号 

\[p(t) \fta P(\omega) = \omega_s \dsum_{-\infty}^{\infty}P_0(n \omega_s) \delta(\omega - n\omega_s)\]

那么
\[f_s(t) \fta F_s(\omega) = \dfrac{1}{2\pi}F(\omega) \otimes \left[\omega_s \dsum_{-\infty}^{\infty}P_0(n \omega_s) \delta(\omega-n\omega_s)\right]\]

化简

\[f_s(t) \fta F_s(\omega) = \dfrac{1}{T_s} \left[ \dsum_{-\infty}^{\infty}P_0(n \omega_s) F(\omega-n\omega_s)\right]\]

\begin{theorem}[时域抽样定理]
  对于原始信号 \(f(t) \fta F(\omega)\) 以及抽样脉冲 \(p(t) \fta P(\omega)\) ,且原始信号带宽有限 \(\abs{\omega} < \omega_m\),
  可以得到抽样频谱 
  \[f_s(t) = p(t) f(t) \fta \dfrac{1}{2 \pi} F(\omega) \otimes P(\omega)\]

  为可以从抽样信号中恢复被抽样信号,在采样间隔 \(\omega_s\) 间,不能发生原信号的重叠,
  即
  \[\omega_s \geq 2 \omega_m\]
  或者可以化为 
  \[\dfrac{2\pi}{T_s} \geq 2 \times 2  \pi f_m\]
  \[T_s \leq \dfrac{1}{2 f_m}\]

  最小的采样频率称为奈奎斯特频率。
\end{theorem}



% End Here

\ifx\mainclass\undefined
\end{document}
\fi 