\ifx\mainclass\undefined
\documentclass[cn,11pt,chinese,black,simple]{../elegantbook}
\usepackage{array}
\newcommand{\ccr}[1]{\makecell{{\color{#1}\rule{1cm}{1cm}}}}


\newcommand{\where}[1]{\Big|_{#1}}
\newcommand{\dd}[1]{\mathrm{d}#1}
\newcommand{\abs}[1]{\left| #1 \right|}
\newcommand{\zt}[1]{\mathscr{Z}[#1]}
\newcommand{\zta}{\xrightarrow{\mathscr{Z}}} 
\newcommand{\lt}[1]{\mathscr{L}[#1]}
\newcommand{\lta}{\xrightarrow{\mathscr{L}}} 
\newcommand{\ft}[1]{\mathscr{F}[#1]}
\newcommand{\fta}{\xrightarrow{\mathscr{F}}} 
\newcommand{\dsum}{\displaystyle\sum}
\newcommand{\aint}{\int_{-\infty}^{+\infty} }

\newcommand{\re}{\operatorname{Re}[\mathrm{s}]} 


\newcommand{\qfig}[2]{\begin{figure}[!htb]
  \centering
  \includegraphics[width=0.6\textwidth]{#1}
  \caption{#2}
\end{figure}}



\renewcommand\arraystretch{1.6}




\setcounter{tocdepth}{3}
\newcommand{\dollar}{\mbox{\textdollar}}
\lstset{
  mathescape = false}

  

\usepackage{shapepar}
\usepackage{longtable}
\usepackage{tikz}
% \usepackage{multirow}
\usetikzlibrary{positioning}
\tikzset{>=stealth}
\newcommand{\tikzmark}[3][]
  {\tikz[remember picture, baseline]
    \node [anchor=base,#1](#2) {#3};}
\usetikzlibrary{graphs}
\begin{document}
\fi 

% Start Here
\chapter{离散时间信号与系统变换域分析}

在时域分析的基础上进行变换域的分析,序列在某种程度上可以看作连续时间信号的特例。依据连续信号分析的套路,进行相应的傅里叶变换,
以及拉普拉斯变换。序列的傅里叶变换是数字信号处理的基本内容,序列拉普拉斯变换则是 \(\mathscr{Z}\)  变换。


\section{\(\mathscr{Z}\) 变换}

\subsection{基本概念}

\begin{definition}[\(\mathscr{Z}\) 变换]
    对于序列 \(x(n)\) ,其 \(\mathscr{Z}\) 双边变换的定义为 
    
    \[X(z) = \zt{x(x)} = \dsum_{n=-\infty}^{+\infty} x(n)z^{-n}\]

    类似拉普拉斯变换,同样定义其单边变换。对于序列 \(x(n)\) ,其 \(\mathscr{Z}\) 单边变换的定义为 
    \[X(z) = \zt{x(x)} = \dsum_{n=0}^{+\infty} x(n)z^{-n}\]
\end{definition}

序列 \(x(n)\) 的 \(\mathscr{Z}\) 变换是复变量 \(z^{-1}\) 的幂级数, \(,\) 其系数是序列 \(x(n)\) 的样值, 即

\[
X(z)=\cdots+x(-2) z^{2}+x(-1) z+x(0)+x(1) z^{-1}+x(2) z^{-2}+\cdots
\]

其实 \(\mathscr{Z}\) 变换可以借助抽样信号的拉普拉斯变换引出。一个信号被理想抽样,得到 

\[x_s(t) = x(t) \delta_T(t) = \dsum_{n=-\infty}^\infty x(nT) \delta(t-nT)\]

进行拉普拉斯变换得到

\[X_s(s) = \dsum_{n=-\infty}^\infty x(nT) e^{-s n T}\]

引入 \(z = e^{s T}\) ,则可以得到

\[X(z) = \dsum_{n=-\infty}^\infty x(n T) z^{-n}\]

离散系统中,采样间隔往往为 \(1\) ,那么 \(z = e^s\)

\[X(z) = \dsum_{n=-\infty}^\infty x(n 1) z^{-n}\]

\subsection{收敛域}

同样是类似于拉普拉斯变换, \(\mathscr{Z}\) 变换同样存在收敛域( ROC ),实际上就是幂级数求和的收敛性问题。

对于变号级数 \(\dsum_{n=-\infty}^{\infty} x(n) z^{-n},\) 满足条件

\[
\lim _{n \rightarrow \infty}\left|\dfrac{x(n+1) z^{-(n+1)}}{x(n) z^{-n}}\right|=\rho
\]

\begin{itemize}
    \item 若 \(\rho < 1 ,\) 则级数绝对收敛
    \item 若 \(\rho>1,\) 则级数发散
    \item 若 \(\rho=1,\) 级数可能收纸也可能发散
\end{itemize}

对于因果序列或 \(n \geq 0\) 序列,收敛域是空心圆。
对于 \(n < 0\) 序列,收敛域是空心圆。
对于双边序列,收敛域是圆环。
对于有限长度序列,分别有 \(n \geq 0\) 、\(n < 0\) 、过 \(0\) 序列,收敛域分别是除去 \(z = 0\),\(z = \infty\) ,\(z = 0\  \text{or} \ \infty\) 。

\subsection{\(z\)平面与\(s\)平面} 

将复数转换为极坐标,原有的以原点为分界变为以单位圆分界。

\[z = e^{(\sigma + j \omega )T} \rightarrow \left\{
\begin{aligned}
    r &= e^{\sigma T} \\
    \theta &= \omega T = \omega \dfrac{2 \pi}{\omega_s}
\end{aligned}\right.
\]

\section{典型的\(\mathscr{Z}\)变换对及性质}


\subsection{变换对}

\begin{longtable}{lll} 
    \caption{\(\mathscr{Z}\)变换对} \\ 
    \toprule
    时域函数 & \(z\)域函数 & ROC \\
    \midrule
    \endfirsthead
    
    \toprule
    时域函数 & \(z\)域函数 & ROC \\
    \midrule
    \endhead 
  
    \hline
    \multicolumn{3}{c}{见下页}\\   \bottomrule
    \endfoot
  
    \bottomrule
    \endlastfoot
    \(\delta(n)\) & 1 & 全平面\\
    \(u(n)\) & \(\dfrac{z}{z-1}\) & \(\abs{z}>1\) \\
    \(a^n u(n)\) & \(\dfrac{z}{z-a}\) & \(\abs{z}>a\) \\
    \(-a^n u(-n-1)\) & \(\dfrac{z}{z-a}\) & \(\abs{z}<a\)\\
    \(\cos(\omega_0 n)u(n)\) & \(\dfrac{z^2-z \cos \omega_0}{z^2 - 2 z \cos\omega_0 + 1}\) & \(\abs{z}>1\) \\
    \(\sin(\omega_0 n)u(n)\) & \(\dfrac{z \sin\omega_0}{z^2 -2 z \cos\omega_0 + 1}\) &  \(\abs{z}>1\) \\

\end{longtable}
  

\subsection{性质}

以下的讨论的基准函数是 \(x(n) \zta X(z)\),并且部分性质对单边双边进行了区分。

\begin{longtable}{llll} 
    \caption{\(\mathscr{Z}\)变换性质} \\ 
    \toprule
    时域函数 & \(z\)域函数 & 原ROC  & 变换后ROC\\
    \midrule
    \endfirsthead
    
    \toprule
    时域函数 & \(z\)域函数 & 原ROC  & 变换后ROC\\
    \midrule
    \endhead 
  
    \hline
    \multicolumn{4}{c}{见下页}\\   \bottomrule
    \endfoot
  
    \bottomrule
    \endlastfoot
    \(x(-n)\) & \(X(z^{-1})\) & \(\alpha < \abs{z} < \beta\) & \(\dfrac{1}{\beta} < \abs{z} < \dfrac{1}{\alpha}\)\\
    \(x(\dfrac{n}{a}),a>0\) & \(X(z^a)\) & \(\alpha < \abs{z} < \beta\) & \(\alpha^{1/a} < \abs{z} < \beta^{1/a}\) \\
    \(x(n \pm m)\) & 双边 \(z^{\pm m}X(z)\) &  \(\alpha < \abs{z} < \beta\) &  \(\alpha < \abs{z} < \beta\) \\
    \(x(n - m)u(n)\) & 单边 \(z^{-m} \left[X(z) + \dsum_{k=-m}^{-1}x(k)z^{-k}\right]\) & \(\abs{z} > a\) & \(\abs{z} > a\) \\
    \(x(n + m)u(n)\) & 单边 \(z^{m} \left[X(z) - \dsum_{k=0}^{m-1}x(k)z^{-k}\right]\) & \(\abs{z} > a\) & \(\abs{z} > a\) \\
    线性性 & & & 原收敛域的交集 \\
    \(n x(n)\) & \(-z \dfrac{\dd{X(z)}}{\dd{z}}\) & \(\alpha < \abs{z} < \beta\) & \(\alpha < \abs{z} < \beta\)\\
    \(n^m x(n)\) & \(\left[-z\dfrac{\dd{}}{\dd{z}}\right]^m X(z)\) & \(\alpha < \abs{z} < \beta\) & \(\alpha < \abs{z} < \beta\)\\
    \(a^n x(n)\) & \(X(\dfrac{z}{a})\) &  \(\alpha < \abs{z} < \beta\) &  \(\alpha < \abs{\dfrac{z}{a}} < \beta\) \\
    \(x_1(n) \otimes x_2(n)\) & \(X_1(z) X_2(z)\) & & 原收敛域交集 \\
    \(x_1(n)x_2(n)\) & \(\dfrac{1}{2 \pi j}\displaystyle\oint_C X_1(\dfrac{z}{v})X_2(v) v^{-1} \dd{v} \)\footnote{其中\(C\)是\(X_1(\dfrac{z}{v})X_2(v)\) 收敛域交集内的逆时针方向围线} & & 收敛域是边界的乘积 \\
\end{longtable}

初值定理 \[\lim _{z \rightarrow \infty} X(z)=\lim _{z \rightarrow \infty} \dsum_{n=0}^{\infty} x(n) z^{-n}=x(0)\]

终值定理 \[\lim_{z\rightarrow 1}(z-1)X(z) = x(\infty)\]
  

\section{逆 \(\mathscr{Z}\) 变换}

根据 \(\mathscr{Z}\) 变换的定义,只要可以将给定的 \(z\) 域函数展开成幂级数的形式,就可以得到原序列。但是往往不是那么容易得到,更多的通过部分分式法确定。

对于有理多项式 \[
    X(z)=\dfrac{B(z)}{A(z)}=\dfrac{b_{m} z^{m}+b_{m-1} z^{m-1}+\cdots+b_{1} z+b_{0}}{a_{n} z^{n}+a_{n-1} z^{n-1}+\cdots+a_{1} z+a_{0}}
\]
对于分解得到的 \(\dfrac{k z}{z-a}\)

\[
\begin{aligned}
    &k a^n u(n), \abs{z} > a\\
    &-k a^n u(-n-1), \abs{z} < a\\
\end{aligned}    
\]

围线积分法 

\[x(n)=\dfrac{1}{2 \pi j } \oint_{c} X(z) z^{m-1} d z\]

该式便是逆变换表达式。\(C\)是 平面上包含 \(X(z)z^{n-1}\)
所有极点的逆时针闭合环路积分路线。

通过留数定理

\[x(n) = \left\{\begin{aligned}
    & \dsum_{\text{polar point in} C} \operatorname{Res}[X(z)z^{n-1}] ,n \geq 0\\
    & \dsum_{\text{polar point out of} C} \operatorname{Res}[X(z)z^{n-1}] ,n < 0\\
\end{aligned}\right.
\]

对于 \(K\) 重极点

\[\operatorname{Res}\left[X(z) z^{n-1}\right]_{z=z_{i}}=\dfrac{1}{(K-1) !} \dfrac{ \dd{} ^{K-1}}{ \dd{} z^{K-1}}\left[\left(z-z_{i}\right)^{K} X(z) z^{n-1}\right]\where{z=z_{i}}\]

\section{\(\mathscr{Z}\) 变换求解系统响应}

LTI离散系统离散时间系统的差分方程一般形式为

\[
\dsum_{k=0}^{N} a_{k} y(n-k)=\dsum_{r=0}^{M} b_{r} x(n-r)
\]

将等式两边取单边\(\mathscr{Z}\) 变换,并利用 \(\mathscr{Z}\) 变换的位移特性

\[
\begin{array}{c}
z[x( n - m ) u ( n )]= z ^{-m}\left[ X ( z )+\dsum_{k=-m}^{-1} x ( k ) z ^{-k}\right] \\
\dsum_{k=0}^{N} a _{k} z ^{-k}\left[ Y ( z )+\dsum_{l=-k}^{-1} y (I) z ^{-1}\right]=\dsum_{r=0}^{M} b _{r} z ^{-r}\left[ X ( z )+\dsum_{m=-r}^{-1} x ( m ) z ^{-m}\right]
\end{array}
\]

那么完全响应的z变换为

\[
Y(z)=\dfrac{\dsum_{r=0}^{M} b_{r} z^{-r}\left[X(z)+\dsum_{m=-r}^{-1} x(m) z^{-m}\right]}{\dsum_{k=0}^{N} a_{k} z^{-k}}-\dfrac{\dsum_{k=0}^{N}\left[a_{k} z^{-k} \cdot \dsum_{l=-k}^{-1} y(l) z^{-l}\right]}{\dsum_{k=0}^{N} a_{k} z^{-k}}
\]

零输入响应的z变换为

\[
Y(z)=-\dfrac{\dsum_{k=0}^{N} a_{k} z^{-k} \cdot \dsum_{l=-k}^{-1} y(l) z^{-l}}{\dsum_{k=0}^{N} a_{k} z^{-k}}
\]

零状态响应的\(\mathscr{Z}\)变换为

\[
Y(z)=\dfrac{\dsum_{r=0}^{M} b_{r} z^{-r}\left[X(z)+\dsum_{m=r}^{-1} x(m) z^{-m}\right]}{\dsum_{k=0}^{N} a_{k} z^{-k}}
\]

当激励 \(x(n)\) 为因果序列时, 系统零状态响应

\[
Y(z)=X(z) \cdot \dfrac{\dsum_{r=0}^{M} b_{r} z^{-r}}{\dsum_{k=0}^{N} a_{k} z^{-k}}
\]

\section{基于 \(\mathscr{Z}\) 变换的系统特性}

\subsection{系统函数}

系统单位样值响应的 \(\mathscr{Z}\) 变换

\[H(z) = \dsum_{n = 0}^\infty h(n) z^{-n}\] 

并满足 \[H(z) = \dfrac{Y(z)}{X(z)} = \dfrac{\dsum_{r=0}^M b_r z^{-r}}{\dsum_{k=0}^N a_k z^{-k}}=\dfrac{\displaystyle\prod_{r=1}^{M}\left(1-z_{r} z^{-1}\right)}{\displaystyle\prod_{k=1}^{N}\left(1-p_{k} z^{-1}\right)}\]

类似拉普拉斯变换,零极点分布可以决定系统的样值响应,可以参见拉普拉斯变换部分的极点分布影响。



\begin{longtable}{ll} 
    \caption{\(\mathscr{Z}\)变换的一阶极点与时域关系} \\ 
    \toprule
    \(X(z)\) 极点位置 & 时域信号性质\\ 
    \midrule
    \endfirsthead
    
    \toprule
    \(X(z)\) 极点位置 & 时域信号性质\\ 
    \midrule
    \endhead 
  
    \hline
    \multicolumn{2}{c}{见下页}\\   \bottomrule
    \endfoot
  
    \bottomrule
    \endlastfoot
    
    \(\abs{z} < 1, \operatorname{Im}[z] = 0\) & 指数衰减 \\
    \(\abs{z} < 1, \operatorname{Im}[z] \neq 0\) & 减幅振荡 \\
    \(\abs{z} = 1, \operatorname{Im}[z] = 0\) & 常数 \\
    \(\abs{z} = 1, \operatorname{Im}[z] \neq 0\) & 等幅振荡 \\
    \(\abs{z} > 1, \operatorname{Im}[z] = 0\) & 指数上升 \\
    \(\abs{z} > 1, \operatorname{Im}[z] \neq 0\) & 增幅振荡 \\


\end{longtable}

若在单位圆上存在高阶极点同样会产生增幅。

\subsection{系统框图}

级联型就是基本的反馈单元。

\qfig{f5.png}{并联框图元件}

\[H_1(z) = \dfrac{1+b_{1,j}z^{-1}}{1-a_{1,j}z^{-1}}\]

\[H_{2}(z)=\dfrac{1+b_{1, i} z^{-1}+b_{2, i} z^{-2}}{1+a_{1, i} z^{-1}+a_{2, i} z^{-2}}\]

\qfig{f6.png}{级联框图元件}

\[H_{1}(z)=\dfrac{1}{1+a_{1, i} z^{-1}} \]

\[ H_{2}(z)=\dfrac{1+b_{1, i} z^{-1}}{1+a_{1, i} z^{-1}+a_{2, i} z^{-2}}\]

\subsection{离散时间系统的特性}

因果性即 \[h(n) = h(n) u(n)\]

因果序列的收敛域为 \(\abs{z} > R\), 那么因果性的充要条件为  \(\abs{z} > R\), 极点分布在一个半径有限的圆中。

稳定性的时域充要条件为 

\[\dsum_{n=-\infty}^{\infty}| h ( n )|< M\]

即 \[H(1) < M\]

系统为稳定的充要条件是:系统函数的收敛域包含
单位圆。

\section{序列傅里叶变换与系统频响}

\subsection{傅里叶变换的性质}

将 \(z\) 表示为极坐标,可发现类似拉普拉斯变换的衰减系数 \(r\)

\[X(z) = X(r e^{j \omega}) = \dsum_{n=-\infty}^{\infty}\left[x(n)r^{-n}\right]e^{-j \omega n}\]

当 \(r = 1\) 时,即单位圆的 \(\mathscr{Z}\) 变换称为系统的傅里叶变换

\[X\left(e^{ j \omega}\right)=\left.X(z)\right|_{z=e^{j \omega}}=\dsum_{n=-\infty}^{\infty} x(n) e^{- j n \omega}\]

按照围线法,其逆变换为 

\[\begin{aligned}
    x(n) &=\dfrac{1}{2 \pi j } \oint_{|z|=1} X(z) z^{n-1} d z \\
    &=\dfrac{1}{2 \pi j } \oint_{|z|=1} X\left(e^{ j \omega}\right) e^{ j n \omega} \cdot e^{- j \omega} d \left(e^{ j \omega}\right) \\
    &=\dfrac{1}{2 \pi j } \int_{-\pi}^{\pi} X\left(e^{ j \omega}\right) e^{ j n \omega} \bullet e^{- j \omega} \bullet j e^{ j \omega} d \omega\\
    &=\dfrac{1}{2 \pi} \int_{-\pi}^{\pi} X\left(e^{ j \omega}\right) e^{j n \omega} d \omega
\end{aligned}\]

并不是任何序列都是存在傅里叶变换。序列存在傅里叶变换的充分条件是绝对可和。

同样存在奇偶虚实性、时移、时域压扩、反褶、线性等性质;
存在序列线性加权、频移、卷积性质。

卷积为 \[x_1(n) \otimes x_2(n) \]


\[
\ft{x_{1}(n) \otimes x_{2}(n)}=X_{1}\left(e^{ j \omega}\right) \cdot X_{2}\left(e^{ j \omega}\right)
\]

\[
\ft{x _{1}( n )  x _{2}( n )}=\dfrac{1}{2 \pi}\left[ X _{1}\left( e ^{ j \omega}\right) \otimes X _{2}\left(e^{ j \omega}\right)\right]
\]

能量守恒定律

\[\dsum_{n=-\infty}^{\infty}| x ( n )|^{2}=\dfrac{1}{2 \pi} \int_{-\pi}^{\pi}\left| X \left( e ^{ j \omega}\right)\right|^{2} d \omega\]

\subsection{系统频响}

稳定系统的傅里叶变换满足 \[H(e^{j\omega}) = H(z)\where{z=e^{j\omega}}\]

物理意义与连续时间系统的一致。

\[
    \begin{array}{l} 
        y ( n )=\left| H \left( e ^{j \omega_{0}}\right)\right| \dfrac{ 1 }{ 2 }\left[ e ^{\left[\left[\varphi\left(\omega_{0}\right)+a_{0} n\right]\right.}- e ^{-\left[\left[\varphi\left(\omega_{0}\right)+\omega_{0} n\right]\right.}\right] \\
        \quad=\left| H \left(e^{ j \omega_{0}}\right)\right| \sin \left[ n \omega_{0}+\varphi\left(\omega_{0}\right)\right]
    \end{array}\]

注意作图法时由于极坐标的周期性,幅频曲线、相频曲线均以\(2 \pi\)为周期。

\section{利用离散系统离散时间系统实现对模拟信号的滤波}

处理实际信号的流程是连续信号离散化,处理之后恢复到连续。即

\[x_c(t) \xrightarrow{\text{A/D转换}} x_d(n) \xrightarrow{\text{数字滤波器}} y_d(n) \xrightarrow{D/A转换} y_c(t)\]

那么具体来说 
\[x_d(n) = x_c(nT)\]

\[y_d(n) = y_c(nT)\]

离散序列由冲激函数串得到

\[\begin{array}{l}
    x_{p}(t)=\dsum_{n=-\infty}^{\infty} x_{c}(n T) \delta(t-n T) \\
    X_{p}( j \omega)=\dsum_{n=-\infty}^{\infty} x_{c}(n T) e^{- j \omega n T}
\end{array}\]

再考虑离散信号的傅里叶变换

\[X_d(e^{j \Omega}) = \dsum_{n=-\infty}^{\infty} x_d(n)e^{-j \Omega n} = \dsum_{n=-\infty}^{\infty} x_c(n T)e^{-j \Omega n}\]

那么 \[X_d(e^{j\Omega}) = X_p(j\Omega /T)\]

抽样关系,\(\omega_s = \dfrac{2\pi}{T}\)

\[X_{p}( j \omega)=\dfrac{1}{T} \dsum_{k=-\infty}^{\infty} X_{c}\left[ j \left(\omega-k \omega_{s}\right)\right]\]

那么

\[X_{d}\left(e^{ j \Omega}\right)=\dfrac{1}{T} \dsum_{k=-\infty}^{\infty} X_{c}[ j (\Omega-2 \pi k) / T]\]

\qfig{f7.png}{频谱关系}

数字滤波器满足

\[Y_{d}\left(e^{ j \Omega}\right)=X_{d}\left(e^{ j \Omega}\right) H_{d}\left(e^{ j \Omega}\right)\]

\[
Y_{d}\left(e^{ j \Omega}\right)=\dfrac{1}{T} \dsum_{k=-\infty}^{\infty} X_{c}[ j (\Omega-2 \pi k) / T] H_{d}\left(e^{ j \Omega}\right)
\]

因为 \(\Omega = \omega T ,\) 故而得到

\[
Y_{p}( j \omega)=\dfrac{1}{T} \dsum_{k=-\infty}^{\infty} X_{c}\left( j \omega-\dfrac{2 \pi k}{T}\right) H_{d}\left(e^{j \omega T}\right)
\]

取 \(\omega_{s}=\dfrac{2 \pi}{T},\) 得到

\[
Y _{d}( j \omega)=\dfrac{ 1 }{ T } \dsum_{k=-\infty}^{\infty} X _{c}\left( j \omega - k \omega_{s}\right) H _{d}\left(e^{ j \omega T}\right)
\]

\[Y_{c}( j \omega)=X_{c}( j \omega) H_{d}\left(e^{j \omega T}\right)\]

即

\[H _{c}( j \omega)=\left\{\begin{array}{cl} 
    H _{d}\left(e^{ j \omega T}\right) & | \omega |<\omega_{s} / 2 \\
    0 & |\omega|>\omega_{2} / 2
    \end{array}\right.\]
% End Here

\ifx\mainclass\undefined
\end{document}
\fi 