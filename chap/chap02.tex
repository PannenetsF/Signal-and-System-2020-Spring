\ifx\mainclass\undefined
\documentclass[cn,11pt,chinese,black,simple]{../elegantbook}
\usepackage{array}
\newcommand{\ccr}[1]{\makecell{{\color{#1}\rule{1cm}{1cm}}}}


\newcommand{\where}[1]{\Big|_{#1}}
\newcommand{\dd}[1]{\mathrm{d}#1}
\newcommand{\abs}[1]{\left| #1 \right|}
\newcommand{\zt}[1]{\mathscr{Z}[#1]}
\newcommand{\zta}{\xrightarrow{\mathscr{Z}}} 
\newcommand{\lt}[1]{\mathscr{L}[#1]}
\newcommand{\lta}{\xrightarrow{\mathscr{L}}} 
\newcommand{\ft}[1]{\mathscr{F}[#1]}
\newcommand{\fta}{\xrightarrow{\mathscr{F}}} 
\newcommand{\dsum}{\displaystyle\sum}
\newcommand{\aint}{\int_{-\infty}^{+\infty} }

\newcommand{\re}{\operatorname{Re}[\mathrm{s}]} 


\newcommand{\qfig}[2]{\begin{figure}[!htb]
  \centering
  \includegraphics[width=0.6\textwidth]{#1}
  \caption{#2}
\end{figure}}



\renewcommand\arraystretch{1.6}




\setcounter{tocdepth}{3}
\newcommand{\dollar}{\mbox{\textdollar}}
\lstset{
  mathescape = false}

  

\usepackage{shapepar}
\usepackage{longtable}
\usepackage{tikz}
% \usepackage{multirow}
\usetikzlibrary{positioning}
\tikzset{>=stealth}
\newcommand{\tikzmark}[3][]
  {\tikz[remember picture, baseline]
    \node [anchor=base,#1](#2) {#3};}
\usetikzlibrary{graphs}
\begin{document}
\fi 

% Start Here

\chapter{连续时间系统的时域分析}

\section{常系数线性微分方程}

常微分方程的求解可以分为以下几个步骤。

\begin{enumerate}
  \item 根据特征方程求解齐次解
  \item 根据激励形式配凑特解,求得对应的系数
  \item 将完全解代入系统初始条件,确定待定系数
\end{enumerate}



\begin{longtable}{ll} 
  \caption{微分方程特解的形式} \\ 
  \toprule
  激励 &  特解形式\\
  \midrule
  \endfirsthead
  
  \toprule
  激励 &  特解形式\\
  \midrule
  \endhead 

  \hline
  \multicolumn{2}{c}{见下页}\\   \bottomrule
  \endfoot

  \bottomrule
  \endlastfoot

  常数\(E\) & 常数\(D\) \\
  \(t^n\) & \(A_1 t^n + A_2 t^{n-1} + \cdots + A_n t + A_{n+1}\) \\
  \(e^{\alpha t}\) & \(A e^{\alpha t}\) \\
  \(\cos (\beta n)\) 或 \(\sin (\beta n)\) & \(P \cos(\beta n) + Q \sin(\beta n)\) \\
  \(t^n e^{\alpha t}\cos (\beta n)\) 或 \(t^n e^{\alpha t}\sin (\beta n)\) & \begin{tabular}[c]{@{}l@{}}\((A_1 t^n + A_2 t^{n-1} + \cdots + A_n t + A_{n+1})e^{\alpha t} \cos(\beta n) +\)\\ \( (B_1 t^n + B_2 t^{n-1} + \cdots + B_n t + B_{n+1})e^{\alpha t}\)\end{tabular} \\

\end{longtable}

\section{常微分方程解的分类}

解的分类。

\begin{itemize}
  \item 自由响应:齐次解
  \item 强迫响应:特解
  \item 暂态响应:时间趋于无穷时解中趋于 0 的部分
  \item 稳态响应:时间趋于无穷时解中不为 0 的部分
  \item 零输入响应:完全由系统储能引起的响应
  \item 零状态响应:完全由激励引起的响应
\end{itemize}



系统时间点的区分。

\begin{itemize}
  \item 起始状态: \(t = 0_-\)
  \item 初始状态: \(t = 0_+\)
\end{itemize}

\section{单位冲激响应}

对于微分方程的单位冲激响应的一般求法,给定一般的系统方程:

\[
\dsum_{i=0}^{n} C_i \dfrac{\dd{^{n-i} r(t)}}{\dd{t^{n-i}}} = \dsum_{j=0}^{m} E_j \dfrac{\dd{^{m-j} e(t)}}{\dd{t^{m-j}}} 
\]

\begin{enumerate}
  \item 将 \(C_0\) 化为 \(1\) ,并且将激励的一侧使用冲激函数 \(\delta(t)\) 完全代替
  \item 为了满足冲激函数的匹配性,那么只有右侧最高阶导数 \(\dfrac{\dd{^n r(t)}}{\dd{t^n}}\) 中含有 \(\delta(t)\)
  \item 那么有\(\dfrac{\dd{^{n-1} r(t)}}{\dd{t^{n-1}}}\left|_{t=0^+}\right. = 1\) 以及 \(\dfrac{\dd{^{i} r(t)}}{\dd{t^{i}}}\left|_{t=0^+}\right. = 0, i < (n-1)\)
  \item 之后根据线性性以及时不变性得到单位冲激响应
\end{enumerate}

\section{卷积计算及性质}

\begin{definition}[卷积]
  对于函数 \(x(t)\), \(y(t)\) 定义其卷积为
  \[s(t) = \int_{-\infty}^{+\infty} x(\tau) y(t-\tau) \dd{\tau}\]
  通常可以表示为
  \[s(t) = x(t) \otimes y(t)\]
  或者
  \[s(t) = x(t) * y(t))\]
\end{definition}

卷积的性质。
\begin{itemize}
  \item 分配率
  \item 交换律
  \item 结合律
  \item 微分积分性质 \[s^{(n-m)}(t) = f_1^{(n)}(t) \otimes f_2^{(-m)}(t) = f_1(t) \otimes f_2'(t)\]
  \item 冲激函数性质 \[f(t-t_1) \otimes \delta(t-t_0) = f(t-t_0-t_1)\] 
        \[f(t) \otimes \delta'(t) = f'(t)\] 
        \[f(t) \otimes \delta^{-1}(t) = \int_{-\infty}^t f(\tau)\dd{\tau}\]
\end{itemize}

\section{时域系统的特性}

LTI 系统的系统性质围绕着单位冲激响应展开。

因果性的\textbf{充要条件}: \[h(t) = h(t) u(t)\]

稳定性的\textbf{充要条件}(绝对可积): \[\int_{-\infty}^{+\infty} |h(t-\tau)| \dd{\tau} \leq N \] 


% End Here

\ifx\mainclass\undefined
\end{document}
\fi 