\ifx\mainclass\undefined
\documentclass[cn,11pt,chinese,black,simple]{../elegantbook}
\usepackage{array}
\newcommand{\ccr}[1]{\makecell{{\color{#1}\rule{1cm}{1cm}}}}


\newcommand{\where}[1]{\Big|_{#1}}
\newcommand{\dd}[1]{\mathrm{d}#1}
\newcommand{\abs}[1]{\left| #1 \right|}
\newcommand{\zt}[1]{\mathscr{Z}[#1]}
\newcommand{\zta}{\xrightarrow{\mathscr{Z}}} 
\newcommand{\lt}[1]{\mathscr{L}[#1]}
\newcommand{\lta}{\xrightarrow{\mathscr{L}}} 
\newcommand{\ft}[1]{\mathscr{F}[#1]}
\newcommand{\fta}{\xrightarrow{\mathscr{F}}} 
\newcommand{\dsum}{\displaystyle\sum}
\newcommand{\aint}{\int_{-\infty}^{+\infty} }

\newcommand{\re}{\operatorname{Re}[\mathrm{s}]} 


\newcommand{\qfig}[2]{\begin{figure}[!htb]
  \centering
  \includegraphics[width=0.6\textwidth]{#1}
  \caption{#2}
\end{figure}}



\renewcommand\arraystretch{1.6}




\setcounter{tocdepth}{3}
\newcommand{\dollar}{\mbox{\textdollar}}
\lstset{
  mathescape = false}

  

\usepackage{shapepar}
\usepackage{longtable}
\usepackage{tikz}
% \usepackage{multirow}
\usetikzlibrary{positioning}
\tikzset{>=stealth}
\newcommand{\tikzmark}[3][]
  {\tikz[remember picture, baseline]
    \node [anchor=base,#1](#2) {#3};}
\usetikzlibrary{graphs}
\begin{document}
\fi 

% Start Here

\chapter{连续时间信号与系统复频域分析}

频域分析法具有鲜明的物理意义,也存在局限性:部分常用的信号不存在傅里叶变换,傅里叶分析计算较为复杂。

拉普拉斯变换是一种应用广泛的数学分析工具,其特点是分析应用较为简单,但物理意义相对模糊。

\section{拉普拉斯变换}

\subsection{定义}

对不存在傅里叶变换的函数增加一个衰减因子使之绝对可积

\[
\begin{aligned}
    F_{1}(\omega)&=\int_{-\infty}^{\infty} e^{-\sigma t} f(t) e^{-j \omega t} d t \\
    &=\int_{-\infty}^{\infty} f(t) e^{-(\sigma+i \omega) t} \mathrm{d} t
\end{aligned}    
\]

记 \(s = \sigma + j \omega\) 令 \(F(s) = F_1(\omega)\) , 拉普拉斯变换记为

\[
F(s)=\int_{-\infty}^{\infty} f(t) e^{-s t} \mathrm{d} t
\]

进行傅里叶逆变换得到

\[
\begin{array}{l}
\begin{aligned}
    e^{-\sigma t} f(t)&=\dfrac{1}{2 \pi} \int_{-\infty}^{\infty} F_{1}(\omega) e^{j \omega t} \mathrm{d} \omega \\
f(t) &=\dfrac{1}{2 \pi} \int_{-\infty}^{\infty} F(s) e^{\sigma t} e^{j \omega t} \mathrm{d} \omega \\
&=\dfrac{1}{2 \pi} \int_{-\infty}^{\infty} F(s) e^{s t} \mathrm{d} \omega
\end{aligned}
\end{array}
\]

又 $s=\sigma+\mathrm{j} \omega$ ,拉普拉斯逆变换记为

\[
f(t)=\dfrac{1}{2 \pi \mathrm{i}} \int_{\sigma-\mathrm{jw}}^{\sigma+j \omega} F(s) e^{s t} \mathrm{d} s
\]

\subsection{单边拉普拉斯变换}

对因果信号 $f(t)=f(t) u(t),$ ,仅考虑 \(t > 0\) 其拉普拉斯变换为

\[
\begin{array}{l}
\begin{aligned}
    
F(s)&=\int_{0}^{\infty} f(t) e^{-s t} \mathrm{d} t \\
f(t)&=\dfrac{1}{2 \pi \mathrm{j}} \int_{\sigma-\mathrm{j} \infty}^{\sigma+j \infty} F(s) e^{s t} \mathrm{d} s, \quad t>0
\end{aligned}
\end{array}
\]

为方便起见,令 $F(s)$ 表示 $f(t)$ 的单边拉普拉斯变换, $F_{B}(s)$ 表示 $f(t)$ 的双边拉普拉斯变换。
单边拉普拉斯变换时,除非明确声明,否则积分下限均取 \(0_-\) ,即

\[
F(s)=\int_{0}^{\infty} f(t) e^{-s t} \mathrm{d} t
\]

\subsection{收敛域}

将使得信号拉普拉斯变换存在的 \(\sigma\) 的取值范围称为该信号拉
普拉斯变换的收敛域,记为ROC (Region of convergence)。为了使每个信号的拉普拉斯变换是唯一的,必须同时给出 \(F(s)\) 与 ROC ,这是和傅里叶变换的明显区别。

收敛域表明了信号进行变换需要的衰减的程度。

在进行单边拉普拉斯变换时,常常忽略 ROC ,因为若是其拉普拉斯变换存在,一定是存在于右半平面,具体位置不影响逆变换的结果。
但是双边拉普拉斯变换\textbf{必须}考虑 ROC ,不同位置的 \(F_B(s)\) 的逆变换是不同的。

\begin{longtable}{ll} 
    \caption{拉普拉斯变换收敛域范围} \\ 
    \toprule
    时域信号 & \(\sigma\) 范围 \\ 
    \midrule
    \endfirsthead
    
    \toprule
    时域信号 & \(\sigma\) 范围 \\ 
    \midrule
    \endhead 
  
    \hline
    \multicolumn{2}{c}{见下页}\\   \bottomrule
    \endfoot
  
    \bottomrule
    \endlastfoot
    \(t > 0\) & \(\sigma > \alpha\) \\
    \(t < 0\) & \(\sigma < \beta\) \\
    \(t \in \mathbb{R}\) & \(\alpha < \sigma < \beta\) \\
    \(n < t < p\) & \(\sigma \in \mathbb{R}\)
  
\end{longtable}

\subsection{傅里叶变换拉普拉斯变换的直观理解}

类似于傅里叶级数,傅里叶变换可以看作是不同幅度 (\(\abs{F(\omega)}\))、不同相位 \(\varphi(\omega)\) 的余弦波或者是指数信号的叠加,是等幅度振荡;拉普拉斯变换则是对不同时间 \(t\) 进行指数衰减的变幅度叠加。

\[
    \begin{aligned}
        f(t)=& \dfrac{1}{2 \pi} \int_{0}^{\infty}\left[F(\sigma-\mathbf{j} \omega) e^{-\mathrm{j} \omega t}+F(\sigma+\mathbf{j} \omega) e^{\mathrm{j} \omega t}\right] e^{\sigma t} \mathbf{d} \omega \\
        =& \dfrac{1}{2 \pi} \int_{0}^{\infty} e^{\sigma t} \mathbf{2}|F(s)| \cos (\omega t+\theta) \mathrm{d} \omega \\
        =&\int_{0}^{\infty} \dfrac{|F(s)| e^{\sigma t} \mathbf{d} \omega}{\pi} \cdot \cos (\omega t+\theta)
    \end{aligned}\]

\section{拉普拉斯变换对}

表格中 \(Re[s] = \sigma\) 。


\begin{longtable}{lll} 
    \caption{拉普拉斯变换对} \\ 
    \toprule
    时域信号 & 复频域信号 & 收敛域\\ 
    \midrule
    \endfirsthead
    
    \toprule
    时域信号 & 复频域信号 & 收敛域\\ 
    \midrule
    \endhead 
  
    \hline
    \multicolumn{3}{c}{见下页}\\   \bottomrule
    \endfoot
  
    \bottomrule
    \endlastfoot
  
    \(e^{-\alpha t}u(t)\) & \(\dfrac{1}{\alpha + s}\) & \(Re[s] > -\alpha\) \\
    \(u(t)\) & \(\dfrac{1}{s}\) & \(Re[s] > 0\) \\
    \(\delta(t)\) & \(1\) & 全 \(s\) 平面 \\
    \(\sin \omega_0 t u(t)\) & \(\dfrac{\omega_0}{s^2 + \omega_0^2}\) & \\
    \(\cos \omega_0 t u(t)\) & \(\dfrac{s}{s^2 + \omega_0^2}\) & \\
    \(e^{-\alpha t}\sin (\omega_0 t) u(t)\) & \(\dfrac{\omega_0}{(s + \alpha)^2 + \omega_0^2}\) & \\
    \(e^{-\alpha t}\cos (\omega_0 t) u(t)\) & \(\dfrac{s + \alpha}{(s + \alpha)^2 + \omega_0^2}\) &\\
    
\end{longtable}

\section{拉普拉斯变换的性质}


以下性质基于 

\[
f(t) \lta F_B(s)    
\]

\[
f(t)u(t) \lta F(s)    
\]

双边拉普拉斯变换的收敛域为 \[\alpha < \re < \beta\]


\begin{longtable}{lll} 
    \caption{拉普拉斯变换性质} \\ 
    \toprule
    名称 & 单边\(\mathscr{L}\)表达形式  & 双边\(\mathscr{L}\)表达形式 \\  
    \midrule
    \endfirsthead
    
    \toprule
    名称 & 单边\(\mathscr{L}\)表达形式  & 双边\(\mathscr{L}\)表达形式 \\  
    \midrule
    \endhead 
  
    \hline
    \multicolumn{3}{c}{见下页}\\   \bottomrule
    \endfoot

    \bottomrule
    \endlastfoot
  
    时移性质 & \(f(t-t_0)u(t-t_0) \lta F(s) e^{-s t_0}\) & \begin{tabular}[c]{@{}l@{}}\(f(t-t_0) \lta F_B(s) e^{-s t_0}\) \\ 收敛域不变\end{tabular}\\
    % 压扩性质 &  &  \(f(a t) \lta \dfrac{1}{\abs{a}} F_B\left(\dfrac{s}{a}\right)\) \\
    压扩性质  & \(f(a t) \lta \dfrac{1}{a} F\left(\dfrac{s}{a}\right)\) \(a > 0\)
                                                                                                                                                                                      & \begin{tabular}[c]{@{}l@{}}\(f(a t) \lta \dfrac{1}{\abs{a}} F_B\left(\dfrac{s}{a}\right)\)\\ 1)  $\alpha>0$ ,$a \alpha<\operatorname{Re}[s]<a \beta$\\ 2) $\alpha<0$ , $a \beta<\operatorname{Re}[\mathrm{s}]<a \alpha$\end{tabular} \\
    线性性质  &                                                                                                                                                                                                         &                                                                                                                                                                                                                                    \\
    时域微分  & \begin{tabular}[c]{@{}l@{}}$\dfrac{\dd{f(t)}}{\dd{t}} \lta s F(s) - f(0_-)$\\ $\dfrac{\dd{^n f(t)}}{\dd{t^n}} \lta s ^{n} F ( s )-\dsum_{k=0}^{n-1} s ^{n-k-1} f ^{(k)}\left( 0 _{-}\right)$\end{tabular} & \begin{tabular}[c]{@{}l@{}}$\dfrac{\dd{^n f(t)}}{\dd{t^n}} \lta s ^{n} F ( s )$\\ 收敛域至少为原收敛域\end{tabular}                                                                                                                           \\
    时域积分  & $\int_{0_{-}}^{t} f(\tau) d \tau \lta \dfrac{f^{(-1)}\left(0_{-}\right)}{s}+\dfrac{F(s)}{s}$                                                                                                              & \begin{tabular}[c]{@{}l@{}}$\int_{0_{-}}^{t} f(\tau) d \tau \lta \dfrac{F(s)}{s}$\\ 收敛域$\max (\alpha, 0)<\operatorname{Re}[s]<\beta$\end{tabular}                                                                                   \\
    \(s\)域平移 & \(e^{s_0 t} f(t) \lta F(s-s_0)\) & \(e^{s_0 t} f(t) \lta F_B(s-s_0)\) \\
    \(s\)域微分 & \(-t f(t) \lta \dfrac{\dd{F(s)}}{\dd{s}}\) &\begin{tabular}[c]{@{}l@{}}\(-t f(t) \lta \dfrac{\dd{F_B(s)}}{\dd{s}}\)\\ 收敛域不变\end{tabular}\\
    \(s\)域积分 & \(\dfrac{f(t)}{t} \lta \int_{s}^{\infty} F(\lambda)\dd{\lambda}\) &\begin{tabular}[c]{@{}l@{}}\(\dfrac{f(t)}{t} \lta \int_{s}^{\infty} F_B(\lambda)\dd{\lambda}\) \\ 收敛域不变\end{tabular}\\
    时域卷积 & \(f_1(t) \otimes f_2(t) \lta F_1(s) F_2(s)\) & \begin{tabular}[c]{@{}l@{}}\(f_1(t) \otimes f_2(t) \lta F_{B,1}(s) F_{B,2}(s)\)\\ 收敛域延伸到并集最大值\end{tabular} \\
    \(s\)域卷积 & \(f_1(t) f_2(t) \lta \dfrac{1}{2 \pi j} F_1(s) \otimes F_2(s)\) & \begin{tabular}[c]{@{}l@{}}\(f_1{t} f_2{t} \lta \dfrac{1}{2 \pi j} F_1(s) \otimes F_2(s)\)\\ 收敛域为边界相加\end{tabular} \\
    初值定理 & \(f(0_+) = \lim_{s\rightarrow +\infty }s F(s)\) & \\
    终值定理 & $\lim _{t \rightarrow \infty} f(t)=\lim _{s \rightarrow 0} s F(s)$ & \\
\end{longtable}


\section{拉普拉斯逆变换}

主要讨论单边变换,介绍部分分式法、留数法,最后介绍双边拉普拉斯变换。

\subsection{部分分式法}

对于有理真分式形式的拉普拉斯变换,

\[F(s)=\dfrac{N(s)}{D(s)}=\dfrac{a_{m} s^{m}+a_{m-1} s^{m-1}+\ldots+a_{1} s+a_{0}}{b_{n} s^{n}+b_{n-1} s^{n-1}+\ldots+b_{1} s+b_{0}}\]

其中\(m\) ,\(n\) 为正整数,且 \(m<n\) ,极点就是使得 \(F(s)\) 为无穷大的 \(s\) 值。零点使 \(F(s) = 0\) 。对其进行因式分解,即可通过变换对进行逆变换。在此仍满足因式分解时,复数极点总是共轭出现的。


\[\begin{array}{l}
    \begin{aligned}
        
        F(s)&=\dfrac{N(s)}{D(s)}=k \dfrac{\left(s-z_{1}\right) \dots\left(s-z_{m}\right)}{\left(s-p_{1}\right) \dots\left(s-p_{n}\right)} \\
        F(s)&=\dfrac{k_{1}}{\left(s-p_{1}\right)}+\ldots+\dfrac{k_{n}}{\left(s-p_{n}\right)} \\
        f(t)&=k_{1} e^{p_{1} t} u(t)+\ldots+k_{n} e^{p_{n} t} u(t)
    \end{aligned}
\end{array}\]

\subsubsection{一阶实数极点}

\[\begin{array}{c}
    \begin{aligned}
        F(s)&=\dfrac{k_{1}}{\left(s-p_{1}\right)}+\ldots+\dfrac{k_{n}}{\left(s-p_{n}\right)} \quad k_{i}=? \\
        F(s)\left(s-p_{i}\right)&=\dfrac{k_{1}}{\left(s-p_{1}\right)}\left(s-p_{i}\right)+\ldots+\dfrac{k_{i}}{\left(s-p_{i}\right)}\left(s-p_{i}\right)+\ldots+\dfrac{k_{n}}{\left(s-p_{n}\right)}\left(s-p_{i}\right) \\
        &=\dfrac{k_{1}}{\left(s-p_{1}\right)}\left(s-p_{i}\right)+\ldots+k_{i}+\ldots+\dfrac{k_{n}}{\left(s-p_{n}\right)}\left(s-p_{i}\right) \\
    \end{aligned}\\
    \vdots \\
    k_{i}=\left[\left(s-p_{i}\right) F(s)\right]_{1 j=p_{i}}
\end{array}\]

\subsubsection{单阶共轭复数极点}

使用实数极点法也可以对复信号进行处理,但是十分繁琐,引出基于正余弦的基本信号,可以极大简化计算。

\subsubsection{重根极点}

\[\begin{array}{c}
    \begin{aligned}
        F(s)&=\dfrac{k\left(s-z_{1}\right) \cdots\left(s-z_{m}\right)}{\left(s-p_{1}\right)^{k}\left(s-p_{2}\right) \cdots\left(s-p_{n-k}\right)} \\
        F(s)&=\dfrac{k_{1 k}}{\left(s-p_{1}\right)^{k}}+\dfrac{k_{1,2}}{\left(s-p_{1}\right)^{k-1}} \cdots+\dfrac{k_{11}}{\left(s-p_{1}\right)}+\dfrac{k_{2}}{s-p_{2}}+\ldots \\ 
        & \quad\quad\quad\quad\quad\quad\quad\quad\quad\quad\quad+\dfrac{k_{n-k}}{s-p_{n-k}} \\
        E(s)&=\dfrac{k_{2}}{s-p_{2}}+\ldots+\dfrac{k_{n-k}}{s-p_{n-k}} \\
        F(s)-E(s)&=\dfrac{k_{1,1}}{\left(s-p_{1}\right)^{k}}+\dfrac{k_{1,2}}{\left(s-p_{1}\right)^{k-1}}+\cdots+\dfrac{k_{1, k}}{\left(s-p_{1}\right)} \\
        \left(s-p_{1}\right)^{k}[F(s)-E(s)]&=k_{1,1}+k_{1,2}\left(s-p_{1}\right)+\ldots+k_{1, k}\left(s-p_{1}\right)^{k-1}
    \end{aligned}
\end{array}\]

通过求导确定同一极点不同阶的系数。

\[k_{1, m}=\left.\dfrac{1}{(m-1) !} \dfrac{ d ^{m-1}}{ d s^{m-1}}\left\{\left(s-p_{1}\right)^{k}[F(s)-E(s)]\right\}\right|_{s=p_{1}}\]

\subsubsection{非真分式的情况}

分解 \(F(s)\) 得到 多项式与真分式两部分,真分式部分通过变换对 \(\delta^{(n)}(t) \lta s^n\) 处理。
\(F(s)\) 含有指数部分,通过频移公式转换。

\subsection{留数法}

逆变换的形式为

\[f(t)=\dfrac{1}{2 \pi j } \int_{\sigma- j \infty}^{\sigma+j \infty} F(s) e^{* \tau} d s, \quad t>0 \]

\qfig{f1.png}{逆\(\mathscr{Z}\)变换积分路径}

转化为

\[f(t)=\dfrac{1}{2 \pi j }\left[\int_{\sigma- j \infty}^{\sigma+ j \infty} F(s) e^{s t } d s+\int_{C_{r}} F(s) e^{s t} d s-\int_{C_{r}} F(s) e^{s t} d s\right]\]

那么转化为环路积分 
\[f(t) = \dfrac{1}{2 \pi j} \oint F(s) e^{st} \dd{s} - \int_{C_r}F(s)e^{st} \dd{s}\]

若是 \[\lim_{\abs{s = r} \rightarrow \infty} \abs{F(s)} = 0\]

那么再结合留数定理 \[f(t) = \dfrac{1}{2 \pi j} \oint F(s) e^{st} \dd{s} =\dsum_{i} \left\{\mathrm{Res}[F(s)e^{st}]\right\}\Big|_{s = p_i}\]

极点满足 

\[\left.\operatorname{Res}\left[F(s) e^{s t}\right]\right|_{s=p_{i}}=\left.\dfrac{1}{(r-1) !}\left\{\dfrac{ d ^{r-1}}{ d s^{r-1}}\left[\left(s-p_{i}\right)^{r} F(s) e^{s t}\right]\right\}\right|_{s=p_{i}}\]


\subsection{双边拉普拉斯变换}

一般来说,可将双边拉氏变换分解为两个类似单边拉氏变换
来处理。在各自的收敛域中的逆变换形式的叠加即可,需要注意不同的收敛域中会出现不同的逆变换形式。

注意,对于左边拉普拉斯变换,相当于进行一次右边变换以及双边反褶。

\section{拉普拉斯变换与傅里叶变换的关系}

\subsection{\(\operatorname{Re}[s] > 0\) 右半平面收敛}

信号不满足绝对可积的条件,
必须衰减才存在傅里叶变换。 只有拉普拉斯变换,傅里叶变换不存在。

\subsection{\(\operatorname{Re}[s] < 0\) 左半平面收敛}

信号绝对可积,\[F(\omega) = F(s)\where{s = j \omega} \]

\subsection{\(\operatorname{Re}[s] = 0 虚轴收敛\)}

\subsubsection{仅一阶虚轴极点}

\[F(s) = \dsum_n \dfrac{k_n}{s - j\omega_n}\]

可得

\[f(t) = \dsum_n k_n e^{j \omega_n t}u(t) \lta F(s)\]

根据傅里叶变换的卷积定理 

\[\begin{aligned}
    e^{j \omega_n t} u(t) \lta &\dfrac{1}{2 \pi} [2 \pi \delta(\omega - \omega_n)] \otimes [\pi \delta(\omega) + \dfrac{1}{j \omega}]\\
    &= \dfrac{1}{j (\omega - \omega_n)} + \pi \delta(\omega - \omega_n)
\end{aligned}\]

那么\[F(\omega) = F(s)\where{s = j\omega} + \dsum_n k_n \pi \delta(\omega-omega_n)\]

\subsubsection{单个多阶虚轴极点}

\[F_{s} = \dfrac{k_0}{(s-j\omega_0)^k}\]

\[\dfrac{k_0}{(k-1)!}t^{k-1}e^{j\omega_0 t}u(t) \lta F(s)\]

同样的进行变换

\[\dfrac{k_0}{(k-1)!}t^{k-1}e^{j\omega_0 t}u(t) \fta \left.F(s)\right|_{s=j \omega}+\dfrac{K_{0} \pi j^{k-1}}{(k-1) !} \delta^{(k-1)}\left(\omega-\omega_{0}\right)\]



\section{拉普拉斯变换求解连续时间系统响应}

\subsection{常微分方程}

对连续时间系统的微分方程使用拉普拉斯变换

\[
\begin{array}{l}
{C}_{0} \dfrac{\mathbf{d}^{n} {r}({t})}{\mathbf{d} t^{n}}+{C}_{1} \dfrac{\mathbf{d}^{n-1} {r}({t})}{\mathbf{d} {t}^{n-1}}+\cdots+{C}_{n-1} \dfrac{\mathbf{d} {r}({t})}{\mathbf{d} t}+{C}_{n} {r}({t}) \\
={E}_{0} \dfrac{\mathbf{d}^{m} {e}({t})}{\mathbf{d} t^{m}}+{E}_{1} \dfrac{\mathbf{d}^{m-1} {e}({t})}{\mathbf{d} t^{m-1}}+\cdots+{E}_{m-1} \dfrac{\mathbf{d} e({t})}{\mathbf{d} t}+{E}_{m} {e}({t})
\end{array}
\]

即

\[
\dsum_{i=0}^{n} C_{i} \dfrac{\mathbf{d}^{n-i} {r}({t})}{\mathbf{d} t^{n-i}}=\dsum_{j=0}^{m} {E}_{j} \dfrac{\mathbf{d}^{m-j} {e}({t})}{\mathbf{d} t^{m-j}}
\]

应用拉普拉斯变换时域微分性质

\[
\left\{\dfrac{\mathrm{d}^{n} f(t)}{\mathrm{d} t^{n}}\right\}\lta s^{n} F(s)-\dsum_{k=0}^{n-1} s^{n-k-1} f^{(k)}\left(0_{-}\right)
\]

\[
    R(s) \dsum_{i=0}^{n} C_{i} s^{n-i}-\dsum_{i=0}^{n} C_{i} \dsum_{k=0}^{n-i-1} s^{n-i-k-1} r_{r}^{(i)}\left(0_{-}\right)=E(s) \dsum_{j=0}^{m} E_{j} s^{m-j}-\dsum_{j=0}^{m} E_{j} \dsum_{i=0}^{m-j-1} s^{m-j-1-1} e^{(i)}\left(0_{-}\right)
\]

可以求解出完全响应的 \(R(s)\),在零状态响应中 \(r^{(k)}(0_-) = 0\),零输入响应中 \(E_i = 0\) 。如果 \(e(t)\) 为因果信号, \(e(t) = e(t)u(t)\) 那么 \(e^{i}(0_-) = 0\) 。

\subsection{电路元件的复频域模型}

基尔霍夫定律 

\[\begin{aligned}
    \dsum I(s) &= 0\\
    \dsum U(s) &= 0
\end{aligned}\]

电阻

\[v_R(t) = R i_R(t) \lta U_R(s) = I_R(s) R\]

电感 

\[v_L(t) = L \dfrac{\dd{i_L(t)}}{\dd{t}} \lta U_L(s) = I_L(s) L s - L i_L(0_-)\]

电容

\[v_C(t) = \dfrac{1}{C}\int_{-\infty}^t i_C(\tau) \dd{\tau} \lta U_C(s) = I_C(s) \dfrac{1}{sC} + \dfrac{1}{s} u_C(0_-)\]


\section{系统函数}

系统的单位冲激相应满足 

\[h(t) \lta H(s)\]

激励信号为 \(e(t)\) 那么响应为 

\[r(t) = e(t) \otimes h(t)\]

那么

\[R(s) = H(s) E(s)\]

系统函数为 
\[H(s) = \dfrac{R(s)}{E(s)}\]



\section{系统结构框图}


\subsection{系统结构}

串联系统 \[H(s) = H_1(s) H_2(s)\]

\qfig{f3.png}{串联结构}

并联系统 \[H(s) = H_1(s) + H_2(s)\]

\qfig{f2.png}{并联结构}

反馈系统 \[H(s) = \dfrac{H_1(s)}{1 + H_1(s) H_2(s) }\]

\qfig{f4.png}{反馈结构}

\subsection{最简系统框图}

因果信号激励时,零状态响应决定的系统函数为 

\[H(s) = \dfrac{R_{zs}}{E(s)} = \dfrac{\dsum_{j=0}^{m}E_j s^{m-j}}{\dsum_{i = 0}^{n} C_i s^{n-i}} = \dfrac{\dsum_{j=0}^{m} E_j s^{m-n-j}}{\dsum_{i=0}^n C_i s^{-i}}\]

将分子分母分别作为一个子系统进行实现。

\section{\(S\) 域零极点分布与时域特性}

极点的分布决定了时域的特性,对于信号而言直接决定了时域信号的形式,对于系统而言决定冲激响应的形式。

\subsection{极点分布}


\begin{longtable}{lll} 
    \caption{拉普拉斯变换的极点与时域关系} \\ 
    \toprule
    \(H(s)\) 极点形式 & 时域信号 & 性质\\ 
    \midrule
    \endfirsthead
    
    \toprule
    \(H(s)\) 极点形式 & 时域信号 & 性质\\ 
    \midrule
    \endhead 
  
    \hline
    \multicolumn{3}{c}{见下页}\\   \bottomrule
    \endfoot
  
    \bottomrule
    \endlastfoot
    
    \(\dfrac{1}{s}\) & u(t) & 常数 \\
    \(\dfrac{1}{s + a}, a > 0\)  & \(e^{-a t} u(t)\) & 指数衰减 \\
    \(\dfrac{1}{s + a}, a < 0\)  & \(e^{-a t} u(t)\) & 指数上升 \\
    \(\dfrac{\omega}{s^2 + \omega^2}\) & \(\sin \omega t u(t)\) & 等幅振荡 \\
    \(\dfrac{\omega}{(s + a)^2 + \omega^2}, a > 0\) & \(e^{-a t} \sin \omega t\) & 衰减振荡 \\
    \(\dfrac{\omega}{(s + a)^2 + \omega^2}, a < 0\) & \(e^{-a t} \sin \omega t\) & 增幅振荡 \\

    \(\dfrac{1}{s^2}\) & \(t u(t)\) & \\
    \(\dfrac{1}{(s+a)^2}, a > 0\) &  \(t e^{-\alpha t} u(t)\) & 上升 \\
    \(\dfrac{1}{(s+a)^2}, a < 0\) &  \(t e^{-\alpha t} u(t)\) & 下降 \\
    \(\dfrac{2\omega s}{(s^2 + \omega^2)^2}\) & \(t \sin t u(t)\) & 增幅振荡 \\


\end{longtable}

\subsection{通过零极点确定系统响应}

激励与系统函数分别表示为

\[E(s)=\dfrac{\prod_{l=1}^{u}\left(s-z_{1}\right)}{\prod_{k=1}^{n}\left(s-P_{k}\right)} \]

\[H(s)=\dfrac{\prod_{j=1}^{m}\left(s-z_{j}\right)}{\prod_{i=1}^{n}\left(s-P_{i}\right)}\]

那么可以分解为

\[R(s) = \dsum_{k=1}^v \dfrac{A_k}{s - p_k} + \dsum_{i = 1}^n \dfrac{A_i}{s - p_i}\]

由激励引起的响应称为强制相应,由系统引起的响应为自由响应。

\section{系统稳定性}

\subsection{稳定性与极点位置}

在时域中,系统的稳定性通过是否绝对可积判断。

系统的稳定性同样取决于极点位置,若是全部极点位于 \(s\) 平面的左半平面那么 可以满足 \(\displaystyle\lim_{t\rightarrow \infty} h(t) = 0)\) ,称之为稳定系统。

同样的若是极点位于右半平面,或者虚轴上存在二阶以上极点, \(\displaystyle\lim_{t\rightarrow \infty} h(t) \rightarrow \infty)\) ,成为不稳定系统。

若是在虚轴上存在一阶零点,由于发生等幅振荡,称为临界稳定系统。

\subsection{稳定性与极点阶数}

系统函数一般表现为

\[\begin{array}{l}
    H(s)=\dfrac{A(s)}{B(s)}=\dfrac{a_{m} s^{m}+a_{m-1} s^{m-1}+\cdots+a_{0}}{b_{n} s^{n}+b_{n-1} s^{n-1}+\cdots+b_{0}} \\
\end{array}\]

在极限情况下 \[\lim _{s \rightarrow \infty} H(s)=\dfrac{a_{m}}{b_{n}} s^{m-n}\]

由于虚轴延伸到无穷远,可以认为无穷远在虚轴上,为了避免不稳定性需要满足 \(m-n \leq 1\) 。

因此 \(m \leq n\) 时,系统稳定,
\(m = n + 1\) 时,临界稳定,
其他情况不稳定。

\subsection{稳定性的判断}

系统函数的一般形式为,

\[\begin{array}{l}
    H(s)=\dfrac{A(s)}{B(s)}=\dfrac{a_{m} s^{m}+a_{m-1} s^{m-1}+\cdots+a_{0}}{b_{n} s^{n}+b_{n-1} s^{n-1}+\cdots+b_{0}} \\
\end{array}\]

对于稳定系统, \(B(s)\) 的系数全为正数,且只能为以下几种情况:

\begin{itemize}
    \item 从最高到最低次幂无缺项
    \item 缺全部偶次项
    \item 缺全部奇次项
\end{itemize}

\subsection{劳斯准则}

对于 \(B(s)\) 的系数列出劳斯阵列,方程式的根全部位于 \(s\) 左半平面的充分且必要
条件为:全部系数 \(b_i\) 为正数,无缺项,阵列中
第一列数字符号相同。

\[\left\{
    \begin{array}{l}
        \begin{aligned}
            &b_n \quad &b_{n-2} \quad &b_{n-4} & \cdots \\
            &b_{n-1} \quad &b_{n-3} \quad &b_{n-5} & \cdots \\
            &c_{n-1} \quad &c_{n-3} \quad &c_{n-5} & \cdots \\ 
            &d_{n-1} \quad &d_{n-3} \quad &d_{n-5} & \cdots \\ 
            &e_{n-1} \quad &e_{n-3} \quad &e_{n-5} & \cdots 
        \end{aligned}
    \end{array}
\right\}\]

其中 

\[\begin{aligned}
    c_{n-1} &=-\dfrac{1}{b_{n-1}}\left|\begin{array}{cc}
    b_{n} & b_{n-2} \\
    b_{n-1} & b_{n-3}
    \end{array}\right| \\
    c_{n-3} &=-\dfrac{1}{b_{n-1}}\left|\begin{array}{cc}
    b_{n} & b_{n-4} \\
    b_{n-1} & b_{n-5}
    \end{array}\right| \\
    d_{n-1} &=-\dfrac{1}{c_{n-1}}\left|\begin{array}{cc}
    b_{n-1} & b_{n-3} \\
    c_{n-1} & c_{n-3}
    \end{array}\right| \\
    d_{n-3} &=-\dfrac{1}{c_{n-1}}\left|\begin{array}{cc}
    b_{n-1} & b_{n-5} \\
    c_{n-1} & c_{n-5}
    \end{array}\right|
\end{aligned}\]

\section{基于复频域与频域的系统特性}

实际上就是使用图解法对系统进行理解,

\[{H}(\mathbf{j} \omega)=\left.{H}(s)\right|_{s=j \omega}=\left.{K} \dfrac{\prod_{j=1}^{m}\left(s-z_{j}\right)}{\prod_{i=1}^{n}\left(s-{P}_{i}\right)}\right|_{s=j \omega}={K} \dfrac{\prod_{j=1}^{m}\left(\mathrm{j} \omega-z_{j}\right)}{\prod_{i=1}^{n}\left(\mathrm{j} \omega-p_{i}\right)}\]

其中的每一项差都可以化为极坐标。

那么就得到幅频特性与相频特性

\[\abs{H(j \omega)} = K \dfrac{N_1 N_2 \cdots N_m}{M_1 M_2 \cdots M_n}\]

\[\varphi(\omega)=\left(\psi_{1}+\psi_{2}+\cdots \psi_{m}\right)-\left(\theta_{1}+\theta_{2}+\cdots \theta_{n}\right)\]



全通网络:幅频特性为常数,零极点逐对关于虚轴对称。

最小相移网络: 所有零点都位于左半平面的网络。

一般网络可以分解为全通网络与
最小相移网络的级连。


\section{周期信号与抽样信号的拉普拉斯变换}

周期信号 \(f(t)\) 的单周期信号 \(f_1(t) = f(t) [u(t) - u(t-T)]\) ,并且 \(f_1(t) \lta F_1(s)\)那么处理成累加


\[f(t) = \dsum_{i = 0}^{\infty} f_1(t-i T)\] 

那么 \[F(s) = F_1(s) \dsum_{n = 0}^{\infty} e^{-nsT} = F_1(s) \dfrac{e^{sT}}{e^{sT}-1}\]


% End Here

\ifx\mainclass\undefined
\end{document}
\fi 