\usepackage{array}
\newcommand{\ccr}[1]{\makecell{{\color{#1}\rule{1cm}{1cm}}}}


\newcommand{\where}[1]{\Big|_{#1}}
\newcommand{\dd}[1]{\mathrm{d}#1}
\newcommand{\abs}[1]{\left| #1 \right|}
\newcommand{\zt}[1]{\mathscr{Z}[#1]}
\newcommand{\zta}{\xrightarrow{\mathscr{Z}}} 
\newcommand{\lt}[1]{\mathscr{L}[#1]}
\newcommand{\lta}{\xrightarrow{\mathscr{L}}} 
\newcommand{\ft}[1]{\mathscr{F}[#1]}
\newcommand{\fta}{\xrightarrow{\mathscr{F}}} 
\newcommand{\dsum}{\displaystyle\sum}
\newcommand{\aint}{\int_{-\infty}^{+\infty} }

\newcommand{\re}{\operatorname{Re}[\mathrm{s}]} 


\newcommand{\qfig}[2]{\begin{figure}[!htb]
  \centering
  \includegraphics[width=0.6\textwidth]{#1}
  \caption{#2}
\end{figure}}



\renewcommand\arraystretch{1.6}




\setcounter{tocdepth}{3}
\newcommand{\dollar}{\mbox{\textdollar}}
\lstset{
  mathescape = false}

  

\usepackage{shapepar}
\usepackage{longtable}
\usepackage{tikz}
% \usepackage{multirow}
\usetikzlibrary{positioning}
\tikzset{>=stealth}
\newcommand{\tikzmark}[3][]
  {\tikz[remember picture, baseline]
    \node [anchor=base,#1](#2) {#3};}
\usetikzlibrary{graphs}